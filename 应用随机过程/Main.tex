\documentclass[11pt]{article}
%基于北京航空航天大学仪器科学与光电工程学院实验报告及课程报告排版得来,类似于毕业论文排版格式
%后续将更新毕业论文排版格式
\usepackage{graphicx,float}%使用图的宏包,使用图的浮动体宏包,引入参数H使图像紧跟当前文字
\usepackage{caption} %使用图表标题的宏包
\usepackage[colorlinks=true,pdfstartview=FitH,%
linkcolor=black,anchorcolor=violet,citecolor=magenta]{hyperref}%加载hyperref宏包,使用超链接
\usepackage{setspace}%用于设置行间距列间距等命令的宏包
\usepackage{array}%设置列表高度宽度的宏包
\usepackage{zhnumber}%使用中文数字编号的宏包
\usepackage{titlesec,titletoc}%使用标题自定义形式的宏包和使用目录自定义形式的宏包
\usepackage{siunitx}%物理学单位宏包
\usepackage{tabularx}%让表格宽度等于页面宽度
\usepackage{makecell}%单个表格单元调整的宏包
\usepackage{subfigure} %%使用子图的宏包
\usepackage[backend=biber,bibstyle=gb7714-1987,%nature,%%加载biblatex宏包,使用参考文献
citestyle=gb7714-1987%,backref=true%%其中后端backend使用biber
,url=false
]{biblatex}%标注(引用)样式citestyle,著录样式bibstyle都采用gb7714-2015样式
% \usepackage{pgfplots}%类似tikz的一个画图库,主要画统计图
\usepackage{../customStyle}
% \usepackage{customFont}%自行编写的字体命令库,基于CJK宏包
% \usepackage{bh_style}%自行编写的风格文件,基于使用习惯和格式要求
% \usepackage{math_formulate}%自行编写的数学公式命令库,基于amsmath宏包
% \usepackage{picture}%集成图形绘制库,主要包括了tikz和pgfplots两大主流宏包
% \usepackage[lite,subscriptcorrection,slantedGreek,nofontinfo]{mtpro2}%使用mathtimepro2商业字体作为数学环境,并不推荐

%biblatex宏包的参考文献数据源加载方式,注意book.bib应当与.tex文件在同一目录下,不然有可能会报错
\addbibresource[location=local]{book.bib}
% % \bibliographystyle{gbt7714-numerical}
%%% 下面的命令重定义页面边距,使其符合中文刊物习惯 %%%%
% \addtolength{\topmargin}{2.5cm}
\setlength{\oddsidemargin}{0.63cm}  % 3.17cm - 1 inch
\setlength{\evensidemargin}{\oddsidemargin}
% \setlength{\textwidth}{14.66cm}
% \setlength{\textheight}{24.00cm}    % 24.62

\graphicspath{{./fig}}

\begin{document}
{
\pagestyle{empty}
\begin{figure}
  \includegraphics{title.jpg}
\end{figure}
\begin{center}

  \begin{figure}[h]
    
    \centering
    \includegraphics[]{title.png}\par
    \vspace{4em}
    \large{\yihao\lishu{2023-2024学年第一学期}}
    \vspace{6em}
  \end{figure}
  
  \large{\erhao\lishu{随机过程理论}}\par
  \large{\erhao\lishu{课程大作业合集}}
  \vspace{8em}
  
  \begin{spacing}{2.0}
    \begin{tabular}{cc}
      
      
      {\xiaoerhao\lishu{班\quad \quad 级}} & {\heiti{\dlmu{SY23173}}}    \\
      {\xiaoerhao\lishu{学\quad \quad 号}} & {\heiti{\dlmu{SY2317301} }} \\
      {\xiaoerhao\lishu{姓\quad \quad 名}} & {\heiti{\dlmu{陈博非} }}       \\
      {\xiaoerhao\lishu{日\quad \quad 期}} & {\heiti{\dlmu{\today} } }   \\
    \end{tabular}
  \end{spacing}
\end{center}
\thispagestyle{empty}
}


\newpage
%手动分页
\pagenumbering{roman}

\setcounter{tocdepth}{3}
%设定目录深度                      
\tableofcontents
%列出目录
\newpage

\pagenumbering{arabic}
\setcounter{page}{1}
\section{第一次大作业}
\subsection{问题描述}
为了生成两个完全不相关的随机信号,现如下图设计了一个信号滤波器,其中输入为一随机信号,两个输出为需要的不相关随机信号。
\begin{figure}[H]
  \centering
  \includegraphics[width=0.8\textwidth]{信号发生器示意图.png}
  \caption{信号滤波器}
  \label{fig:信号滤波器}
\end{figure}

\subsection{设计思路}
首先从原理上讨论应该如何设计两个线性系统,考虑两个输出信号之间的互相关函数。即有:
\begin{equation}
  \begin{aligned}
    R_{XY}(t_1,t_2) & =E\{y_1(t_1)y_2(t_2)\}                                                                                               \\
                    & =E\{[x_1(t)*h_1(t)][x_2(t)*h_2(t)]\}                                                                                 \\
                    & =E\Big\{\int_{-\infty}^{+\infty}x_1(u)h_1(t_1-u)\mathrm{d}u\int_{-\infty}^{+\infty}x_2(v)h_2(t_2-v)\mathrm{d}v\Big\} \\
                    & =\int_{-\infty}^{+\infty}\int_{-\infty}^{+\infty}h_1(t_1-u)h_2(t_2-v)E[x_1(u)x_2(v)]\mathrm{d}u\mathrm{d}v           \\
                    & =\int_{-\infty}^{+\infty}\int_{-\infty}^{+\infty}R_x(u,v)h_1(t_1-u)h_2(t_2-v)\mathrm{d}u\mathrm{d}v                  \\
                    & =R_x(t_1,t_2)*h_1(t_1)*h_2(t_2)
  \end{aligned}
\end{equation}
可见,两个输出信号之间的互相关函数相当于对输入信号的自相关函数进行两次卷积,如果输入的函数是平稳的,两个输出的信号之间是联合平稳的,则可以使用功率谱密度函数表示上述结果,得到以下形式:
\begin{equation}
  \begin{aligned}
    S_{XY}(\omega) & =S_X(\omega)H_1(\omega)H_2^*(\omega)
  \end{aligned}
\end{equation}
可见,当两个输出信号之间的互相关函数为零时,功率谱密度函数也为0,则需要两个线性系统的频率响应函数在每一个频率上均是乘积为0的。换言之,需要设计两个线性系统,使得两个系统的频率响应函数不为零的位置相互错开,即可得到两个输出不相关的信号。

考虑最简单的情况,即设计一个低通滤波器和一个高通滤波器,使得两个滤波器的截止频率相互错开,其在频率上的乘积即可以满足近似为0。工程上,认为阻带内的衰减小于-15dB即可认为是0,滤波器的截止频率定义为幅频增益为-3dB处的频率。\cite{signal_system}。
\subsection{设计过程}
由于问题没有给任何信号的具体形式,这里也没有必要定义某一个具体的截止频率,不妨设低通滤波器的截止频率(角频率)是$\omega_1$,高通滤波器的截止频率是$\omega_2$,则有:
\subsubsection{低通滤波器}
如下图,低通滤波器的幅频特性为:
\begin{figure}[H]
  \centering
  \includegraphics[width=0.8\textwidth]{低通滤波器.png}
  \caption{低通滤波器幅频特性说明}
  \label{fig:低通滤波器}
\end{figure}

可将低通滤波器的频率特性写为:
\begin{displaymath}
  H_1(j\omega)=\frac{1}{\displaystyle 1+j\frac{\omega}{\omega_L}}
\end{displaymath}
其中$\omega_L$为截止频率。
\subsection{高通滤波器}
同理,对于高通滤波器,其幅频特性为:
\begin{figure}[H]
  \centering
  \includegraphics[width=0.8\textwidth]{高通滤波器.png}
  \caption{高通滤波器幅频特性说明}
  \label{fig:高通滤波器}
\end{figure}
可将高通滤波器的频率特性写为:
\begin{displaymath}
  H_2(j\omega)=\frac{1}{\displaystyle 1+j\frac{\omega_H}{\omega}}
\end{displaymath}
其中$\omega_H$为高通截止频率。
\subsubsection{设计结果}
需要注意的是,首先应当保证高通滤波器的截止频率高于低通滤波器的截止频率,即有:
\begin{equation*}
  \omega_2>\omega_1
\end{equation*}

其次是为了保证在任意频率处的幅值增益都小于-15dB,则需要在两个幅频特性相等的位置处,幅值增益的和也小于-15dB,即有:
\begin{align}
  \left\{
  \begin{aligned}
     & 20\lg\mmode{\frac{1}{\displaystyle 1+j\frac{\omega}{\omega_L}}}=20\lg\mmode{\frac{1}{\displaystyle 1+j\frac{\omega_H}{\omega}}}      \\\\
     & 20\lg\mmode{\frac{1}{\displaystyle 1+j\frac{\omega}{\omega_L}}}+20\lg\mmode{\frac{1}{\displaystyle 1+j\frac{\omega_H}{\omega}}}<-15
  \end{aligned}
  \right.
\end{align}
对于某一个给定的截止频率,可得两幅频特性曲线之交点频率,以及两个截止频率需要满足的条件为:
\begin{align*}
  \left\{
  \begin{aligned}
     & \omega=\sqrt{\omega_H\omega_L}                        \\\\
     & \frac{\omega_L^2}{\omega_H^2+\omega_L^2}<10^{-0.375}
  \end{aligned}
  \right.
\end{align*}
因此可以设计得到在任意频率点处,均可以以-15dB负增益。则从输入到输出的功率谱密度函数为:
\begin{align*}
  S_{XY}(\omega)=\frac{S_X(\omega)}{\displaystyle 1+\frac{\omega_H}{\omega_L}+j\circbrac{-\frac{\omega_H}{\omega}+\frac{\omega}{\omega_L}}}
\end{align*}
其中,一个比较简单的实现形式为:
\begin{align*}
  H_1(j\omega)=\frac{1}{\displaystyle 1+j\frac{\omega}{\omega_L}} \implies h_1(t)=\left\{
  \begin{aligned}
     & e^{\displaystyle -\frac{1}{\omega_L}t} & ,  t>0    \\
     & 0                                      & ,  t\le0
  \end{aligned}
  \right.
\end{align*}

\begin{align*}
  H_2(j\omega)=\frac{1}{\displaystyle 1+j\frac{\omega_H}{\omega}} \implies h_1(t)=\left\{
  \begin{aligned}
     & 0                                               & , t>0    \\
     & \delta(t)-e^{\displaystyle \frac{1}{\omega_H}t} & , t\le0
  \end{aligned}
  \right.
\end{align*}
这里的高通滤波器是物理不可实现的,因此常常使用巴特沃斯滤波器或者切比雪夫滤波器形式的函数实现高通滤波,而不是使用指数函数的形式,给定截止频率和阻带衰减,即可以设计出对应的高通滤波器。
\section{第二次大作业}
\subsection{问题描述}
分析一个应用泊松随机过程的实例。

\newpage
\printbibliography[heading=bibliography,title=参考文献]
\end{document}
