\documentclass[11pt]{article}
%基于北京航空航天大学仪器科学与光电工程学院实验报告及课程报告排版得来,类似于毕业论文排版格式
%后续将更新毕业论文排版格式
\usepackage{graphicx,float}%使用图的宏包,使用图的浮动体宏包,引入参数H使图像紧跟当前文字
\usepackage{caption} %使用图表标题的宏包
\usepackage[colorlinks=true,pdfstartview=FitH,%
linkcolor=black,anchorcolor=violet,citecolor=magenta]{hyperref}%加载hyperref宏包,使用超链接
\usepackage{setspace}%用于设置行间距列间距等命令的宏包
\usepackage{array}%设置列表高度宽度的宏包
\usepackage{zhnumber}%使用中文数字编号的宏包
\usepackage{titlesec,titletoc}%使用标题自定义形式的宏包和使用目录自定义形式的宏包
\usepackage{siunitx}%物理学单位宏包
\usepackage{tabularx}%让表格宽度等于页面宽度
\usepackage{makecell}%单个表格单元调整的宏包
\usepackage{subfigure} %%使用子图的宏包
\usepackage[backend=biber,bibstyle=gb7714-1987,%nature,%%加载biblatex宏包,使用参考文献
citestyle=gb7714-1987%,backref=true%%其中后端backend使用biber
,url=false
]{biblatex}%标注(引用)样式citestyle,著录样式bibstyle都采用gb7714-2015样式
% \usepackage{pgfplots}%类似tikz的一个画图库,主要画统计图
\usepackage{../customStyle}
% \usepackage{customFont}%自行编写的字体命令库,基于CJK宏包
% \usepackage{bh_style}%自行编写的风格文件,基于使用习惯和格式要求
% \usepackage{math_formulate}%自行编写的数学公式命令库,基于amsmath宏包
% \usepackage{picture}%集成图形绘制库,主要包括了tikz和pgfplots两大主流宏包
% \usepackage[lite,subscriptcorrection,slantedGreek,nofontinfo]{mtpro2}%使用mathtimepro2商业字体作为数学环境,并不推荐

%biblatex宏包的参考文献数据源加载方式,注意book.bib应当与.tex文件在同一目录下,不然有可能会报错
\addbibresource[location=local]{book.bib}
% % \bibliographystyle{gbt7714-numerical}
%%% 下面的命令重定义页面边距,使其符合中文刊物习惯 %%%%
% \addtolength{\topmargin}{2.5cm}
\setlength{\oddsidemargin}{0.63cm}  % 3.17cm - 1 inch
\setlength{\evensidemargin}{\oddsidemargin}
% \setlength{\textwidth}{14.66cm}
% \setlength{\textheight}{24.00cm}    % 24.62

\graphicspath{{./fig}}

\begin{document}
{
\pagestyle{empty}
\begin{figure}
  \includegraphics{title.jpg}
\end{figure}
\begin{center}

  \begin{figure}[h]
    
    \centering
    \includegraphics[]{title.png}\par
    \vspace{4em}
    \large{\yihao\lishu{2023-2024学年第一学期}}
    \vspace{6em}
  \end{figure}
  
  \large{\erhao\lishu{计算机视觉多视图几何}}\par
  \large{\erhao\lishu{课程大作业}}
  \vspace{8em}
  
  \begin{spacing}{2.0}
    \begin{tabular}{cc}
      
      
      {\xiaoerhao\lishu{班\quad \quad 级}} & {\heiti{\dlmu{SY23173}}}    \\
      {\xiaoerhao\lishu{学\quad \quad 号}} & {\heiti{\dlmu{SY2317301} }} \\
      {\xiaoerhao\lishu{姓\quad \quad 名}} & {\heiti{\dlmu{陈博非} }}       \\
      {\xiaoerhao\lishu{日\quad \quad 期}} & {\heiti{\dlmu{\today} } }   \\
    \end{tabular}
  \end{spacing}
\end{center}
\thispagestyle{empty}
}


\newpage
%手动分页
\pagenumbering{roman}

\setcounter{tocdepth}{3}
%设定目录深度                      
\tableofcontents
%列出目录
\newpage

\pagenumbering{arabic}
\setcounter{page}{1}
\section{简介}
\subsection{问题背景}
在常规的计算机视觉应用场景中,通过摄像机等成像设备得到的场景图像往往是射影图像,其中的点、线等测度关系均由于摄像机等因素的射影变换而产生失真,不能用于直接测量和计算。因此,如何从射影失真的图像中恢复测量关系,是解决计算机视觉领域后续问题的基础,本文将讨论这一问题的通常解决办法,以及影响该问题的因素以及如何选取最佳的解决方案。
\subsection{问题描述}
\begin{itemize}
  \item 问题一:根据现场拍摄的一幅图像,自行选取关键点,使用二维平面的仿射变换校正为仿射图像;
  \item 问题二:讨论如何选取校正关键点,不同校正关键点的选取对校正结果的影响,并使用齐次坐标等数学工具证明为什么某些选取方式是不行的,而某些能行;
  \item 问题三:使用仿真的方法,画棋盘格等生成虚拟场景,使用相机模型对虚拟场景成像,应用不同的方法对仿真照片处理,根据处理结果对不同的校正方法进行评价。
\end{itemize}

\section{仿射变换校正}
将经过射影变换失真后的图像恢复至可测量,需要恢复至相似变换,可以采用的方法诸如的先使用无穷远直线恢复仿射变换,然后使用虚圆点恢复相似变换;抑或是使用一个圆的像曲线和无穷远直线的交点直接恢复至相似变换;本文将要讨论的方法是,不通过仿射变换和虚圆点的选取,直接使用正交直线从射影变换恢复至相似变换。
\subsection{数学描述}
设原始图像的绝对对偶二次曲线为$C_\infty^*$,原始图像上任意两条直线段为$\mathbf{l}_i=(l_i^1,l_i^2,l_i^3)^\textrm{T}$和$\mathbf{m}_i=(m_i^1,m_i^2,m_i^3)^\textrm{T}$,任意某个点为$\mathbf{a}=(x_i,y_i,z_i)$,经过一个变换矩阵为$H$的射影变换后,变换后的图像上的对偶二次曲线为$C_\infty^{*'}$,对应的直线段为$\mathbf{l}_i'=(l_i^{1'},l_i^{2'},l_i^{3'})^\textrm{T}$和$\mathbf{m}_i'=(m_i^{1'},m_i^{2'},m_i^{3'})^\textrm{T}$,像平面上的某个点为$\mathbf{a}'=(x_i',y_i',z_i')$,需要找出满足这些条件的像直线段$\mathbf{l}_i'$和$\mathbf{m}_i'$,使得变换矩阵$H$能够被计算出,并用于射影变换图像的校正。
\subsection{问题结论}
这里先给出该问题的结论,这样的直线至少需要10条,分为五对相互正交,各个正交直线对彼此之间不重叠、不平行,抑或是由这样的五对正交直线生成的\textbf{双线性矩阵}满秩。

式中的双线性运算措辞借鉴自\cite{hartley2003multiple},原书中对双线性乘积计算的定义为:
\begin{align}
  \label{eq:bilinear}
  \mathbf{(l|m)}\triangleq \det(A,B,\~A,\~B)=l_{12}m_{34}+l_{34}m_{12}+l_{13}m_{42}+l_{42}m_{13}-l_{14}m_{23}-l_{23}m_{14}
\end{align}
使用双线性乘积计算定义了两条直线的相交关系。在本文中,将使用到这一性质。在原始图像中,经过点$\mathbf{a}$和点$\mathbf{b}$的直线可以使用Plücker矩阵表示为:
\begin{align}
  \label{eq:plucker}
  \mathbf{l}_{ab}=\mathbf{a}\mathbf{b}^\textrm{T}-\mathbf{b}\mathbf{a}^\textrm{T}
\end{align}
在二维图像中,Plücker矩阵退化为两个齐次坐标点的向量积,即:
\begin{align}
  \label{eq:plucker2d}
  \mathbf{l}_{ab}=\mathbf{a}\times\mathbf{b}
\end{align}

\section{校正关键点选取}

\section{仿真实验}
\newpage
\printbibliography[heading=bibliography,title=参考文献]
\end{document}
