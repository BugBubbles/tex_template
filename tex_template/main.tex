\documentclass[11pt]{article}
%第二十届中国研究生数学建模竞赛论文模板
\usepackage{graphicx,float}%使用图的宏包,使用图的浮动体宏包,引入参数H使图像紧跟当前文字
\usepackage{caption} %使用图表标题的宏包
\usepackage[colorlinks=true,pdfstartview=FitH,%
linkcolor=black,anchorcolor=violet,citecolor=magenta]{hyperref}%加载hyperref宏包,使用超链接
\usepackage{setspace}%用于设置行间距列间距等命令的宏包
\usepackage{array}%设置列表高度宽度的宏包
\usepackage{zhnumber}%使用中文数字编号的宏包
\usepackage{titlesec,titletoc}%使用标题自定义形式的宏包和使用目录自定义形式的宏包
\usepackage{siunitx}%物理学单位宏包
\usepackage{tabularx}%让表格宽度等于页面宽度
\usepackage{makecell}%单个表格单元调整的宏包
\usepackage{booktabs}%给表格添加框线的命令
\usepackage{subfigure} %%使用子图的宏包
\usepackage[backend=biber,bibstyle=gb7714-1987,%nature,%%加载biblatex宏包,使用参考文献
citestyle=gb7714-1987%,backref=true%%其中后端backend使用biber
,url=false
]{biblatex}%标注(引用)样式citestyle,著录样式bibstyle都采用gb7714-2015样式
% \usepackage{pgfplots}%类似tikz的一个画图库,主要画统计图
\usepackage{customFont}%自行编写的字体命令库,基于CJK宏包
\usepackage{customFormat}%自行编写的风格文件,基于使用习惯和格式要求
\usepackage{customMath}%自行编写的数学公式命令库,基于amsmath宏包
\usepackage{customPicture}%集成图形绘制库,主要包括了tikz和pgfplots两大主流宏包
% \usepackage[lite,subscriptcorrection,slantedGreek,nofontinfo]{mtpro2}%使用mathtimepro2商业字体作为数学环境,并不推荐

%biblatex宏包的参考文献数据源加载方式,注意book.bib应当与.tex文件在同一目录下,不然有可能会报错
\addbibresource[location=local]{reference.bib}
% % \bibliographystyle{gbt7714-numerical}
%%% 下面的命令重定义页面边距,使其符合中文刊物习惯 %%%%
% \addtolength{\topmargin}{2.5cm}
\setlength{\oddsidemargin}{0.63cm}  % 3.17cm - 1 inch
\setlength{\evensidemargin}{\oddsidemargin}
% \setlength{\textwidth}{14.66cm}
% \setlength{\textheight}{24.00cm}    % 24.62

\graphicspath{{./fig}}
\begin{document}
\begin{center}

  \begin{figure}[H]
    \includegraphics[width=0.2\textwidth]{标题图1.png}
    \hspace{0mm}
    \includegraphics[width=0.2\textwidth]{标题图2.png}
    \hspace{3mm}
    \includegraphics[width=0.14\textwidth]{标题图3.png}
    \hspace{4mm}
    \includegraphics[width=0.16\textwidth]{标题图4.png}
    \hspace{4mm}
    \includegraphics[width=0.18\textwidth]{标题图5.png}
  \end{figure}
  \large{\xiaoerhao\hwxw{\textbf{中国研究生创新实践系列大赛}}}\\
  \large{\erhao\hwxw{\textbf{“华为杯”第二十届中国研究生}}}\\
  \large{\erhao\hwxw{\textbf{数学建模竞赛}}}
  \vspace{8em}

  \begin{spacing}{2.0}
    \begin{tabular}{m{0.2\textwidth}<{\centering}m{0.6\textwidth}<{\centering}}
      {\xiaoerhao\hwzs{学\quad \quad 校}} & {\xiaoerhao\hwzs{123456班}}                                                            \\
      \cmidrule(r){1-2}
      {\xiaoerhao\hwzs{参赛队号} }          & {\xiaoerhao\hwzs{12345678} }                                                          \\
      \cmidrule(r){1-2}
      {\xiaoerhao\hwzs{队员姓名}}           & \begin{tabular}{m{0.1\textwidth}<{\centering}m{0.45\textwidth}<{\centering}}
        \xiaoerhao{1.} & \xiaoerhao\hwzs{葛瑄} \\    \cmidrule(r){1-2}
        \xiaoerhao{2.} &  \xiaoerhao\hwzs{张严} \\    \cmidrule(r){1-2}
        \xiaoerhao{3.} &  \xiaoerhao\hwzs{陈博非} \\
                                          \end{tabular}
      \\
      \bottomrule
    \end{tabular}

  \end{spacing}
\end{center}
\thispagestyle{empty}
\newpage

\begin{center}
  \large{\xiaoerhao\hwxw{\textbf{中国研究生创新实践系列大赛}}}\\
  \large{\erhao\hwxw{\textbf{“华为杯”第二十届中国研究生}}}\\
  \large{\erhao\hwxw{\textbf{数学建模竞赛}}}
  \vspace{2em}
  \par
\end{center}

\begin{tabular}{m{0.2\textwidth}<{\centering}m{0.7\textwidth}<{\centering}}
  \xiaoerhao\lishu{ 题\quad\quad 目:} & \dlmu{\xiaoerhao\lishu{梯度下降法和遗传算法}} \\
  &\dlmu{\xiaoerhao\lishu 在复值矩阵分解的应用}\\

\end{tabular}
\begin{abstract}
  本文主要使用了梯度下降法对矩阵函数进行优化。\\
  当矩阵取值\cite{telescope}
\end{abstract}
{\xiaoerhao\lishu 关键词:这里是关键词}
\begin{section}{问题重述}
 这里是第一章的正文
 \begin{subsection}{问题的提出}
   这里是第一章第一节的正文
   \begin{figure}[H]
     \centering
     \includegraphics[width=0.4\textwidth]{标题图2.png}
     \caption{这里是图片的标题}
     \label{fig:这里是图片的标签}
   \end{figure}
   \begin{subsubsection}{  问题的背景}
     这里是第一章第一节第一小节的正文
   \end{subsubsection}
   \begin{table}[H]
     \centering
     \caption{红外成像芯片参数表}
     \renewcommand{\arraystretch}{1.5}
     \label{table:红外成像芯片参数}
     \begin{tabular}{|m{0.3\textwidth}<{\centering}|m{0.3\textwidth}<{\centering}|m{0.3\textwidth}<{\raggedright\arraybackslash}|}
       \hline
       项目     & 参数                               & \makecell*[c]{备注}                                       \\\hline
       探测器类型  & 非制冷氧化钒                           &                                                         \\\hline
       响应波段   & $8\sim 14\unit{\um}$             &                                                         \\\hline
       红外分辨率  & $ 640 \times 480$                &                                                         \\\hline
       成像单元尺寸 & $12\unit{\um}\times12\unit{\um}$ &                                                         \\\hline
       NETD   & $50\unit{mK}@F1.0/50\unit{\Hz}$  & {50mK表示噪声等效温差为50mK,F1.0指相对孔径为1.0,50Hz表示此参数是帧频50Hz下测得的 } \\\hline
       帧频     & $50\unit{\Hz}$                   &                                                         \\\hline
     \end{tabular}
   \end{table}
 \end{subsection}
 这里是一个数学公式,示例为薛定谔方程
 \begin{equation}
   \label{eq:薛定谔方程}
   \mathrm{i}\hbar\frac{\partial}{\partial t}\Psi(\mathbf{r},t)=\left[\frac{-\hbar^2}{2\mu}\nabla^2+V(\mathbf{r},t)\right]\Psi(\mathbf{r},t)
 \end{equation}
 这里是一个行内数学公式,示例为质能方程$E=mc^2$。
\end{section}
\printbibliography[heading=bibliography,title=\centering 参考文献]
\end{document}