\documentclass{article}
%基于北京航空航天大学仪器科学与光电工程学院实验报告及课程报告排版得来,类似于毕业论文排版格式
%后续将更新毕业论文排版格式
\usepackage{graphicx,float}%使用图的宏包,使用图的浮动体宏包,引入参数H使图像紧跟当前文字
\usepackage{caption} %使用图表标题的宏包
\usepackage[colorlinks=true,pdfstartview=FitH,%
linkcolor=black,anchorcolor=violet,citecolor=magenta]{hyperref}%加载hyperref宏包,使用超链接
\usepackage{setspace}%用于设置行间距列间距等命令的宏包
\usepackage{array}%设置列表高度宽度的宏包
\usepackage{zhnumber}%使用中文数字编号的宏包
\usepackage{titlesec,titletoc}%使用标题自定义形式的宏包和使用目录自定义形式的宏包
\usepackage{siunitx}%物理学单位宏包
\usepackage{tabularx}%让表格宽度等于页面宽度
\usepackage{makecell}%单个表格单元调整的宏包
\usepackage{subfigure} %%使用子图的宏包
\usepackage{dirtree}
\usepackage[backend=biber,%nature,%%加载biblatex宏包,使用参考文献
gbnamefmt=quanpin,%将文献作者姓氏区分大小写
bibstyle=gb7714-2005,%加载参考文献样式表
maxbibnames=3,
%nature,%%加载biblatex宏包,使用参考文献
citestyle=gb7714-2005,%加载引注样式
%backref=false,%不显示引用文献的页码
gblocal=gb7714-2005,%使中英文献各自输入中英文的“和”与“等”
url=false,%注意,report类型文档类下,url和报告时间是必须的
doi=false,%不显示网址和doi
gbpunctin=false,%不显示文献中的//析出符号
gbmedium=false,%不显示OL符号
mergedate=none
]{biblatex}%标注(引用)样式citestyle,著录样式bibstyle都采用gb7714-2015样式
% \usepackage{pgfplots}%类似tikz的一个画图库,主要画统计图
\usepackage{../customStyle}
\addbibresource[location=local]{bibliography.bib}
\graphicspath{{./fig}}

\fancyhf{} 
% 页眉页脚设置
\lhead{陈博非}
\chead{陨石坑识别}
\rhead{\today}
\cfoot{\thepage}
\rfoot{}
\lfoot{}
% 加入页眉
\pagestyle{fancy}
% 表格行间距调整为1.5
\renewcommand{\arraystretch}{1.0}

\begin{document}
\section{关于椭圆}
由于陨石坑的边缘始终是当作椭圆,因此在计算边缘、计算不确定度时大量使用到了椭圆的一般式和齐次矩阵,这里详细记录一遍。定义椭圆的一般式定义为:
\begin{equation}
  Ax^2+2Bxy+Cy^2+2Dx+2Ey+F=0\label{eq:ellipse}
\end{equation}\par
务必注意一般式定义中含有系数2,因此在椭圆的齐次矩阵中取消该系数:
\begin{equation}
  \mathbf{C}=\begin{bmatrix}
    A&B&D\\
    B&C&E\\
    D&E&F
  \end{bmatrix}
\end{equation}\par
\subsection{椭圆参数计算}
通过一般式求解椭圆的参数过程烦琐,但是可以算,解出椭圆的中心为:
\begin{equation*}
  \begin{cases}
    x_0=\frac{CD-BE}{B^2-AC}\\
    y_0=\frac{AE-BD}{B^2-AC}
  \end{cases}
\end{equation*}\par
椭圆的长短轴长度为:
\begin{equation*}
  \begin{cases}
    a^2=\frac{2(AE^2+CD^2+BF^2-2BDE-ACF)}{(B^2-AC)(\sqrt{(A-C)^2+4B^2}-A-C)}\\
    b^2=\frac{2(AE^2+CD^2+BF^2-2BDE-ACF)}{(B^2-AC)(-\sqrt{(A-C)^2+4B^2}-A-C)}
  \end{cases}\triangleq\begin{cases}
    a^2=\frac{2N_a}{D_a}\\
    b^2=\frac{2N_b}{D_b}
  \end{cases}
\end{equation*}\par
注意在求解过程中,经常因为椭圆被压扁为直线,导致解出来的$a^2$或$b^2$为负数,此时应当将这种情况舍去。
\subsection{不确定度}
计算椭圆的不确定度才是令人发指的,首先考察不确定度的由来,例如,对于同一个椭圆的边缘点,现有$n$个观测点$(x_i,y_i,1)$,则可利用奇异值分解算出椭圆一般式参数,如下式:
\begin{equation*}
  \begin{bmatrix}
    x_1^2&2x_1y_1&y^2_1&2x_1&2y_1&1\\
    x_2^2&2x_2y_2&y^2_2&2x_2&2y_2&1\\
    \vdots&\vdots&\vdots&\vdots&\vdots&\vdots\\
    x_n^2&2x_ny_n&y^2_n&2x_n&2y_n&1
\end{bmatrix}
  \begin{bmatrix}
    A\\B\\C\\D\\E\\F
  \end{bmatrix}=\mathbf{AX}=\mathbf{0}\implies\mathbf{U}\mathbold{\Sigma}\mathbf{ V}^\top=\mathbf{A}\implies\mathbf{X}=\mathbf{V}_3
\end{equation*}\par
即矩阵$\mathbf{V}$的最后一列是$\mathbf{X}$的解。此时求解不确定度,可将各个点的观测值当作对同一个椭圆的独立重复试验,且为了简化求解过程,将$\mathbf{X}$的六个分量当作服从相同噪声分布,且彼此独立,此时可根据最小二乘公式一步求出不确定度:
\begin{equation}
  \mathbf{D}=\frac{\mathbf{v^\top v}}{n-6}\mathbf{VV^\top}
  \label{eq:SVD}
\end{equation}\par
式中的$\mathbf{v=AX}$是正规方程的残差,虽然这样计算存在较多的简化和理想化结果,但是并不太影响最终的结果。接下来,可以从公式\ref{eq:SVD}中一步计算出六个参数的不确定度,即各自主对角线元素的算术平方根,如下式:
\begin{equation*}
  u(A)=\sqrt{D_{11}},u(B)=\sqrt{D_{22}},u(C)=\sqrt{D_{33}},u(D)=\sqrt{D_{44}},u(E)=\sqrt{D_{55}},u(F)=\sqrt{D_{66}}
\end{equation*}\par
利用本式,推导陨石坑中心点的不确定度,开根号即可:
\begin{equation}
  \begin{aligned}
    u^2(x_0)&=\left[\frac{CD-BE}{(B^2-AC)^2}\right]^2u^2(A)+\left[\frac{EB^2-AEC-2BCD}{(B^2-AC)^2}\right]^2u^2(B)\\
    &+\left[\frac{DB^2-ABE}{(B^2-AC)^2}\right]^2u^2(C)+\left[\frac{C}{(B^2-AC)}\right]^2u^2(D)+\left[\frac{B}{(B^2-AC)}\right]^2u^2(E)\\
    u^2(y_0)&=\left[\frac{EB^2-CBD}{(B^2-AC)^2}\right]^2u^2(A)+\left[\frac{DB^2-ACD-2ABE}{(B^2-AC)^2}\right]^2u^2(B)\\
    &+\left[\frac{AE-BD}{(B^2-AC)^2}\right]^2u^2(C)+\left[\frac{B}{(B^2-AC)}\right]^2u^2(D)+\left[\frac{A}{(B^2-AC)}\right]^2u^2(E)\\
  \end{aligned}
\end{equation}\par
同样地,推导椭圆半长轴和半短轴的不确定度,\textbr{易得}:
\begin{equation}
  \left\{
  \begin{aligned}
    u^2(a)&=\frac{1}{a}\sum_{x}\left(\frac{\pdiff{N_a}{x}D_a-\pdiff{D_a}{x}N_a}{D^2_a}\right)^2u^2(x)\\
    u^2(b)&=\frac{1}{b}\sum_{x}\left(\frac{\pdiff{N_b}{x}D_b-\pdiff{D_b}{x}N_b}{D^2_b}\right)^2u^2(x)\\
  \end{aligned}\right.
\end{equation}\par
其中,$x$遍历$A,B,C,D,E,F$,分别有:
\begin{align*}
  &\left\{
    \begin{aligned}
      \pdiff{N_a}{A}&=E^2-CF,\pdiff{D_a}{A}=-C\left[\sqrt{(A-C)^2+4B^2}-A-C\right]+(B^2-AC)\left[\frac{A-C}{\sqrt{(A-C)^2+4B^2}}-1\right]\\
      \pdiff{N_b}{A}&=E^2-CF,\pdiff{D_b}{A}=-C\left[-\sqrt{(A-C)^2+4B^2}-A-C\right]+(B^2-AC)\left[-\frac{A-C}{\sqrt{(A-C)^2+4B^2}}-1\right]\\ 
    \end{aligned}
  \right.\\
  &\left\{
    \begin{aligned}
      \pdiff{N_a}{B}&=F^2-2DE,\pdiff{D_a}{B}=2B\sqrt{(A-C)^2+4B^2}+(B^2-AC)\left[\frac{4B}{\sqrt{(A-C)^2+4B^2}}\right]\\
      \pdiff{N_b}{B}&=F^2-2DE,\pdiff{D_b}{B}=-2B\sqrt{(A-C)^2+4B^2}-(B^2-AC)\left[\frac{4B}{\sqrt{(A-C)^2+4B^2}}\right]\\ 
    \end{aligned}
  \right.\\
  &\left\{
    \begin{aligned}
      \pdiff{N_a}{C}&=D^2-AF,\pdiff{D_a}{C}=-A\left[\sqrt{(A-C)^2+4B^2}-A-C\right]+(B^2-AC)\left[\frac{C-A}{\sqrt{(A-C)^2+4B^2}}-1\right]\\
      \pdiff{N_b}{C}&=D^2-AF,\pdiff{D_b}{C}=-A\left[-\sqrt{(A-C)^2+4B^2}-A-C\right]-(B^2-AC)\left[\frac{C-A}{\sqrt{(A-C)^2+4B^2}}-1\right]\\ 
    \end{aligned}
  \right.
\end{align*}
\begin{align*}
  &\left\{
    \begin{aligned}
      \pdiff{N_a}{D}&=2CD-2BE,\pdiff{D_a}{D}=0\\
      \pdiff{N_b}{D}&=2CD-2BE,\pdiff{D_b}{D}=0\\ 
    \end{aligned}
  \right.\\
  &\left\{
    \begin{aligned}
      \pdiff{N_a}{E}&=2AE-2BD,\pdiff{D_a}{E}=0\\
      \pdiff{N_b}{E}&=2AE-2BD,\pdiff{D_b}{E}=0\\ 
    \end{aligned}
  \right.\\
  &\left\{
    \begin{aligned}
      \pdiff{N_a}{F}&=2BF-AC,\pdiff{D_a}{F}=0\\
      \pdiff{N_b}{F}&=2BF-AC,\pdiff{D_b}{F}=0\\ 
    \end{aligned}
  \right.
\end{align*}\par
代入可得长短轴的不确定度。
\subsection{Hanak描述子}
Hanak\cite{hanakCraterIdentificationAlgorithm2010}提出的不变量描述子定义为:
\begin{equation}
  \left[\frac{2r_1}{l_1},\frac{2r_2}{l_1},\frac{2r_3}{l_1},\cos\alpha_1,\cos\alpha_2,I_\mathrm{CW/CCW}\right]
\end{equation}\par
其中,需要事先保证按内角从大到小,排序为1、2和3,才能计算以下不变量,基于该计算结果,可求出当前不变量的不确定度,如下:
\begin{equation}
  \begin{aligned}
    u\left(\frac{2r_1}{l_1}\right)&=\frac{2}{l_1}\sqrt{\left[\frac{u(r_1)}{r_1}\right]^2+\left(\frac{u(l_1)}{l_1}\right)^2}\\
    u\left(\frac{2r_2}{l_1}\right)&=\frac{2}{l_1}\sqrt{\left[\frac{u(r_2)}{r_2}\right]^2+\left(\frac{u(l_1)}{l_1}\right)^2}\\
    u\left(\frac{2r_3}{l_1}\right)&=\frac{2}{l_1}\sqrt{\left[\frac{u(r_3)}{r_3}\right]^2+\left(\frac{u(l_1)}{l_1}\right)^2}\\
    u(\cos\alpha_1)&=\frac{l_1}{l_2l_3}\sqrt{\cos^2\alpha_3u^2(l_2)+\cos^2\alpha_2u^2(l_3)+u^2(l_1)}\\
    u(\cos\alpha_2)&=\frac{l_2}{l_1l_3}\sqrt{\cos^2\alpha_3u^2(l_1)+\cos^2\alpha_1u^2(l_3)+u^2(l_2)}\\
    u(I_\mathrm{CW/CCW})&=0
  \end{aligned}
\end{equation}\par
其中,半径可近似认为由长短两轴平均而得:
\begin{equation*}
  r=\frac{a+b}{2}\implies u(r)=\frac{1}{2}\sqrt{u^2(a)+u^2(b)}
\end{equation*}\par
而最大边长$l_1$可由选定的两个顶点2和3计算所得,因此,其不确定度为:
\begin{equation*}
  u(l_1)=\frac{1}{l_1}\sqrt{u^2(x_1)+u^2(y_1)+u^2(x_2)+u^2(y_2)}
\end{equation*}\par
\subsection{互逆描述子}
互逆描述子由Quan\cite{quanInvariantsPairConics1992}等人提出,其定义为:
\begin{equation*}
  I_1 = \mathrm{tr}(\mathbf{C}_1^{-1}\mathbf{C}_2)\quad I_2 = \mathrm{tr}(\mathbf{C}_2^{-1}\mathbf{C}_1)
\end{equation*}\par

\newpage
\printbibliography[heading=bibliography,title=参考文献]
\end{document}
