\documentclass{article}
%基于北京航空航天大学仪器科学与光电工程学院实验报告及课程报告排版得来,类似于毕业论文排版格式
%后续将更新毕业论文排版格式
\usepackage{graphicx,float}%使用图的宏包,使用图的浮动体宏包,引入参数H使图像紧跟当前文字
\usepackage{caption} %使用图表标题的宏包
\usepackage[colorlinks=true,pdfstartview=FitH,%
linkcolor=black,anchorcolor=violet,citecolor=magenta]{hyperref}%加载hyperref宏包,使用超链接
\usepackage{setspace}%用于设置行间距列间距等命令的宏包
\usepackage{array}%设置列表高度宽度的宏包
\usepackage{zhnumber}%使用中文数字编号的宏包
\usepackage{titlesec,titletoc}%使用标题自定义形式的宏包和使用目录自定义形式的宏包
\usepackage{siunitx}%物理学单位宏包
\usepackage{tabularx}%让表格宽度等于页面宽度
\usepackage{makecell}%单个表格单元调整的宏包
\usepackage{subfigure} %%使用子图的宏包
\usepackage{dirtree}
\usepackage[backend=biber,%nature,%%加载biblatex宏包,使用参考文献
gbnamefmt=quanpin,%将文献作者姓氏区分大小写
bibstyle=gb7714-2005,%加载参考文献样式表
maxbibnames=3,
%nature,%%加载biblatex宏包,使用参考文献
citestyle=gb7714-2005,%加载引注样式
%backref=false,%不显示引用文献的页码
gblocal=gb7714-2005,%使中英文献各自输入中英文的“和”与“等”
url=false,%注意,report类型文档类下,url和报告时间是必须的
doi=false,%不显示网址和doi
gbpunctin=false,%不显示文献中的//析出符号
gbmedium=false,%不显示OL符号
mergedate=none
]{biblatex}%标注(引用)样式citestyle,著录样式bibstyle都采用gb7714-2015样式
% \usepackage{pgfplots}%类似tikz的一个画图库,主要画统计图
\usepackage{../customStyle}
\usepackage{algorithm, algpseudocode}
\addbibresource[location=local]{bibliography.bib}
\graphicspath{{./fig}}
\begin{document}
基于CO-DETR和地形NMS的中小型导航陨石坑目录构建方案
\section{Abstract}
\label{sec:abstract}
\section{Introduction}
深空探测是当前航天领域的研究热点,自主导航能力是深空探测航天器的必要能力之一。地外天体表面丰富的陨石坑能够作为有效路标,可用于航天器绕飞、着陆等过程的自主导航,因此,获取高精度、详尽的地外天体陨石坑目录,尤其是中小型(10-100m)陨石坑目录,是进行深空探测后续任务的基础。
嫦娥五号于2020年12月1日成功着陆于月球虹湾地区43.06°N,51.92°W,围绕其着陆点,中国航天局和NASA等机构产出大量深空遥感图像,便于构建嫦娥五号着陆点附近的陨石坑目录。深空遥感图像类型包括光学遥感图像、红外遥感图像、数字高程地图以及数字地形地图等,通过遥感图像作光度校正、几何校正等预处理手段,可得到正射的遥感图像,此时陨石坑形式均呈现为规则的圆形外形,便于通过陨石坑检测方法提取其中心和半径,提取陨石坑的算法被称为“陨石坑检测算法CDA”。
根据提取陨石坑的技术路径,可将陨石坑检测方法分为基于手工设计的特征和基于数据驱动的特征两种类型。基于手工设计的特征包括Hough变换特征\cite{emamiCraterDetectionUsing2019}、阴影-高亮特征\cite{maassEdgefreeScalePose2011,maassRobustApproximationImage2016}、Canny边缘\cite{emamiCraterDetectionUsing2019}、地形路径特征\cite{wangNovelApproachMultiscale2022}、小波模板\cite{diMachineLearningApproach2014,wangActiveMachineLearning2019}和匹配模板\cite{bandeiraDetectionSubkilometerCraters2012}等,通过对特定区域或特征遥感图像的统计分析,手工设计的特征可取得较高的陨石坑检测召回率与精确率,但是这类方法必须针对不同区域、不同遥感图像修改模型参数,在处理较多图像数据时很不方便,且手工设计的指标往往还需要针对不同特征设计不同的检测后端算法,特征的使用不具有通用性。深度学习和计算机视觉的发展改变了传统的特征设计方式,基于数据驱动的特征能够在数据集上自主学习、更新,并保证在训练集上取得最佳性能,是当前设计自动化陨石坑检测所采用的主流特征。根据深度学习采用的特征提取器类型,可将特征分为基于VGG和ResNet\cite{wangCraterIDNetEndendFully2018,yangLunarImpactCrater2020,chenCraterDetectionRecognition2021}、基于UNet\cite{silburtLunarCraterIdentification2019,songEffectiveLunarCrater2020}、基于Transformer\cite{daiBoostingCraterDetection2023}、基于SNN\cite{zhaEnergyefficientCratersDetection2024}等,其输出的高维向量特征较为复杂,难以可视化,往往通过目标检测头网络完成陨石坑中心坐标与宽高范围的边界框回归计算,由此得到完整的深度学习陨石坑检测网络,如基于单阶段模型的YOLOv3\cite{fairweatherAutomaticMappingSmall2022}、YOLOv5\cite{liuTwostageAdaptiveNetwork2024}、YOLOv7\cite{daiBoostingCraterDetection2023}、SSD\cite{silvestriniOpticalNavigationLunar2022};基于两阶段模型的Faster RCNN\cite{yangLunarImpactCrater2020,yangCraterDANetConvolutionalNeural2022,liuIdentificationLunarCraters2024}、CraterIDNet\cite{wangCraterIDNetEndendFully2018}和HRFPNet\cite{yangHighresolutionFeaturePyramid2022}、LCDNet\cite{miaoLCDNetInnovativeNeural2024};基于无锚点模型的CenterNet\cite{zhangAutomaticDetectionSmallscale2024};基于端到端检测模型的Crater-DETR\cite{guoCraterDETRNovelTransformer2024}。这些目标检测网络能够在几何校正、光度校正后的正射遥感图像上取得极高的召回率和精确率(超过90\%),通过诸如特征金字塔FPN\cite{linFeaturePyramidNetworks2017}和卷积注意力CBAM\cite{wooCBAMConvolutionalBlock2018}等技巧,能够提升检测网络对小尺寸陨石坑的检测表现,并在不同的光照条件下取得稳定的结果。但是,现有模型仍然对小陨石坑缺乏敏感性。

遥感正射图像坐标系通过坐标系变换和投影变换可转换至月球经纬度坐标,因此利用转换公式,可以将不同遥感图像上的陨石坑统一至相同的经纬坐标系下,借助边界框非极大抑制算法,合并相同位置处重复出现的陨石坑检测结果,由此可得到全体候选陨石坑组成的候选陨石坑目录,需要对候选陨石坑目录作检验筛选,排除其中错误的陨石坑,才能得到最终的陨石坑目录输出。

现有的陨石坑检验方案中,较为可靠、应用较多的是专家检验,例如用于陨石坑检测与检验工具CraterTools\cite{kneisslMapprojectionindependentCraterSizefrequency2011},能够在人工标注选点基础上计算陨石坑圆参数,并自动折算至经纬坐标系下,此外还能通过使用不同投影方法和多点测量验证陨石坑尺寸测量结果。基于CraterTools,结合遥感光学图像与遥感DEM,Head\cite{headGlobalDistributionLarge2010}、Povilaitis\cite{povilaitisCraterDensityDifferences2018}、Robbins\cite{robbinsNewGlobalDatabase2019}和Yang\cite{yangLunarImpactCrater2020}等人提出了全月陨石坑目录;Jia\cite{jiaCatalogueImpactCraters2020}、Qian\cite{qianCopernicanaged200Ma2021}和Bo\cite{boCatalogueMeterscaleImpact2022}等人提出了嫦娥五号附近的显著型和小型陨石坑目录。专家检验费时费力,难以保证复现一致性。Wang\cite{wangImprovedGlobalCatalog2021}等人利用陨石坑地形数据设计验证算法,能够有效提升检验效率并保证可靠性。但是,这种方法不能筛去错误的陨石坑,且在边缘破碎的风化陨石坑处容易产生异常结果。
本文需要解决检测与检验两方面问题的自动化,同时保证检验算法的可靠性。在检测上,采用目前先进的Co-DETR算法输出边界框检测结果。在检验上,事先将DEM与遥感光学图像配准,从Bo\cite{boCatalogueMeterscaleImpact2022}等人提供的小型陨石坑目录中获取单个陨石坑的三维点,设计检验判别网络DNet,将判别网络输出的判别概率和检测网络输出的置信度以贝叶斯公式合并后得到置信概率,并由此将全部候选陨石坑检验后取得陨石坑目录。本文的工作贡献主要有:
\begin{itemize}
  \item 将CO-DETR用于陨石坑检测,在现有陨石坑检测数据集上取得最佳表现。
  \item 提出了一种基于地形数据的非极大抑制算法,精确确定候选陨石坑尺度与位置。
  \item 提供了嫦娥五号着陆点附近详细的中小型陨石坑目录。
\end{itemize}
\section{Related Works}
\label{sec:related_works}
正射遥感图像上的陨石坑检测多采用计算机视觉领域的目标检测算法。按检测头网络的构成基本单元,可分为基于CNN的检测网络和基于Transformer的检测网络。
\subsection{陨石坑检测}
\subsubsection{基于CNN}
主流的CNN检测网络以RCNN和YOLO为代表,其中以Faster RCNN\cite{renFasterRCNNRealtime2015}为代表的双阶段网络因其具备的区域建议网络RPN,在陨石坑检测任务精确率、召回率表现上更佳。Wang\cite{wangCraterIDNetEndendFully2018}等人将Faster RCNN的全连接回归头修改为全卷积回归头,得到CraterIDNet,该网络减少了参数量,但是缺乏对小尺度陨石坑的检测能力。Chen\cite{chenCraterDetectionRecognition2021}等人在CraterIDNet基础上增加FPN,增强了尺度敏感性,但是需要依赖稠密的锚点,训练过程较为复杂。Liu\cite{liuIdentificationLunarCraters2024}和Miao\cite{miaoLCDNetInnovativeNeural2024}等人不修改Faster RCNN的结构,分别使用基于小样本学习和自适应NMS的训练策略,利用FPN结构增强小尺寸陨石坑检测能力,实现在不同数据集条件下的学习和复用,但是实验表明Faster RCNN模型泛化能力不佳,在与源数据集差异较大的测试集上检测召回率很低。Yang\cite{yangLunarImpactCrater2020}和Yang\cite{yangCraterDANetConvolutionalNeural2022}等人考虑使用迁移学习和Cycle-GAN\cite{zhuUnpairedImageimageTranslation2017}跨域学习增强泛化能力,但是无疑使训练策略异常复杂,不利用于实际使用。
单阶段网络如YOLO系列、SSD等模型使用预设的锚点代替了RPN,提升了实时检测能力,但是在小尺寸陨石坑检测上表现不佳。Dai\cite{daiBoostingCraterDetection2023}等人提出用ViT\cite{dosovitskiyImageWorth16x162021}融合两种数据模态的YOLOv7网络,该网络改善了单模态检测表现,但是在整体表现上仍然弱于双阶段网络。Liu\cite{liuTwostageAdaptiveNetwork2024}等人结合归一化自适应(Normalization-Based Attention Module)、双阶段监督和跨域训练增强YOLOv5,改善了网络精确率,但是召回率大幅下降。此外另有多位学者直接应用YOLOv3\cite{fairweatherAutomaticMappingSmall2022}、YOLOv8\cite{wangCatalogueImpactCraters2024}和SSD\cite{silvestriniOpticalNavigationLunar2022}进行陨石坑检测,均在小尺寸陨石坑上表现不佳。基于热力图的无锚点检测网络CenterNet\cite{zhangAutomaticDetectionSmallscale2024,yangTopographicKnowledgeawareNetwork2024}尚不及有锚点的单阶段和双阶段网络。
\subsubsection{基于Transformer}
构成CNN网络的核心运算卷积,仅具备有限的感受野,其对图像的整体把握能力不足。Transformer提出的注意力机制\cite{vaswaniAttentionAllYou2017}弥补了这一缺点,综合CNN和Transformer搭建的DETR\cite{carionEndtoendObjectDetection2020}开创端到端检测的全新领域,但是DETR难以收敛、并未在性能上显著超越CNN基础的检测模型,仍然有较大的提升空间。因此,Deformable DETR\cite{zhuDeformableDETRDeformable2021}用弹性注意力机制取代基于MLP的注意力机制,引入类似锚点机制的焦点偏离,加快了检测网络的收敛。DINO\cite{zhangDINODETRImproved2023}在DAB-DETR\cite{liuDABDETRDynamicAnchor2022}提出的锚点查询框和DN-DETR\cite{liDNDETRAccelerateDETR}提出的带噪训练策略基础上,将解码器层的级联关系由一次回顾修改s为两次回顾,进一步提升了模型的表现性能和收敛速度。针对训练中正样本数量远远少于负样本,导致注意力学习缓慢的问题,CO-DETR\cite{zongDETRsCollaborativeHybrid2023}加入多组协作分支,增加编码器中产生的正样本查询值,有效提升了DETR系列检测模型的表现能力。鉴于DETR系列模型在目标检测领域取得先进水平,Crater-DETR\cite{guoCraterDETRNovelTransformer2024}在RT-DETR\cite{zhaoDETRsBeatYOLOs2024}结构基础上增加了噪声训练和小样本IoU损失函数,显著改善了小陨石坑的检测能力,是目前首个将DETR模型用于陨石坑检测的算法,但是在召回率表现上仍然有提升空间。
\subsection{后处理算法}
检测算法所得候选陨石坑需经过后处理算法才能形成陨石坑目录。常用的后处理算法包括非极大抑制、检验确认等。
\subsubsection{非极大抑制}
普通NMS采用简单的IoU计算,保留相同标签下最高置信度的边界框,但是其存在诸如误删、耗时以及IoU阈值不易确定等问题。Soft NMS\cite{bodlaSoftNMSImprovingObject2017}用指数衰减被抑制候选框的置信分数,Softer NMS\cite{heSofterNMSRethinkingBounding2018}在其基础上加入增加对定位置信度的计算、以对应顶点加权合并作为抑制结果,提升了边界框的定位精度,但是Softer NMS需要修改检测头的回归输出,在训练阶段需要回传梯度。Adaptive NMS\cite{liuAdaptiveNMSRefining2019}针对不同的物体密度自适应设计抑制阈值,改善了稠密重叠物体的检测表现,Adaptive NMS也需要训练。针对NMS耗时较长的问题,Matrix NMS\cite{wangSOLOSimpleFramework2021}用矩阵乘法替代循环加快了计算过程,但是需要占用较多的存储资源。由于陨石坑具备圆形外表,Liu\cite{liuIdentificationLunarCraters2024}等人用圆形框取代方形框计算IoU,但是在重叠陨石坑条件下存在误删情况。Miao\cite{miaoLCDNetInnovativeNeural2024}等人将Adaptive NMS应用于重叠陨石坑的检测,减少了重叠情况对NMS的影响。
\subsubsection{检验确认}
大多数候选陨石坑经检验确认后可用于制作目录,现有检验算法大多采用专家手工检验。其中,Yang\cite{yangLunarImpactCrater2020}等人仅在遥感光学图像上确认;Head\cite{headGlobalDistributionLarge2010}、Povilaitis\cite{povilaitisCraterDensityDifferences2018}、Robbins\cite{robbinsNewGlobalDatabase2019}、Jia\cite{jiaCatalogueImpactCraters2020}、Qian\cite{qianCopernicanaged200Ma2021}和Bo\cite{boCatalogueMeterscaleImpact2022}等人结合地形数据和光学图像共同确认。目前仅有少数学者研究使用算法进行陨石坑检验确认,Wang\cite{wangImprovedGlobalCatalog2021}等人以候选陨石坑中心沿八个等分方向取最高点作为真实边缘点重新拟合圆,以此作为陨石坑真实边缘,该算法仅考虑了单个候选陨石坑的地形数据,不能排除错误的检测结果。
\section{Method}
\label{sec:method}
构建陨石坑目录需要陨石坑检测和检验两大步骤,本章节着重介绍用于目标检测的CO-DINO网络构造原理、训练策略以及用于陨石坑检验的DEM-NMS算法。
\subsection{协作分支}
极度不平衡的正负查询比例,使得基于DETR框架检测模型在训练过程中,有效的一一对应匹配梯度难以回传至深层网络,即Encoder上,导致Encoder层内的注意力模块难以被正确监督学习,降低了其检测性能。Co-DETR\cite{zongDETRsCollaborativeHybrid2023}通过引入大量的协作分支,在单个协作分支上使用随机选定的边界框分配方式,为每一个ground truth边界框分配多个协作框,形成类似于锚点机制的边界框分配方式,极大丰富了正查询样本的占比,从而保证正负查询比例协调。Co-DETR仅使用了最浅层的特征图,即最粗糙的特征图,经过多次降采样和卷积运算后,按照simple feature pyramids\cite{liExploringPlainVision2022}的构成方法产生特征金字塔,作为其协作分支的输入。协作分支各自的损失函数定义为:
\begin{equation}
  \mathcal{L}_\text{co}(y,\hat{y}) = \mathcal{L}_i(c^\text{pos}_i,\hat{c}_i^{pos})+\mathcal{L}_i(c^\text{neg}_i, \hat{c}_i^\text{neg})
\end{equation}\par
其中,$y=(c,\mathbf{b})=(c,(x,y,w,h))\in\mathbb{R}\times\mathbb{R}^{4}$为ground truth,包括了标签值与边界框参数,$\hat{y}$为对应预测值,$c$为GT类别,$\hat{c}$为预测类别,$\mathbf{b}$为GT边界框,$\mathbf{\hat{b}}$为预测边界框,上标pos和neg分别表示第i个分配方式决定下的正负样本。协作分支的总损失函数定义为:
\begin{equation}
  \mathcal{L}_\text{co} = \sum_{i=1}^{N}\mathcal{L}_i(y,\hat{y})
  \label{eq:co-loss}
\end{equation}\par
其中,$N$为协作分支的数量。实际上,不仅是分配的正样本标签,正样本的边界框坐标回归也应当参与至协作分支的损失函数中,以进一步增强协作分支监督学习对Encoder的作用。这里选用简单的Smooth L1 loss作为边界框回归损失函数,将式\ref{eq:co-loss}改写为:
\begin{equation}
  \mathcal{L}_\text{co} = \sum_{i=1}^{N}\left[\mathcal{L}_i(c^\text{pos}_i,\hat{c}_i^{pos})+\mathcal{L}_i(c^\text{neg}_i, \hat{c}_i^\text{neg})+\lambda\mathcal{L}_{\text{box},i}(\mathbf{b}^\text{pos}_i,\mathbf{\hat{b}}_i^\text{pos})\right]
  \label{eq:co-loss-reg}
\end{equation}\par
\subsection{小尺度损失函数}
基于DETR框架的检测模型,其训练的损失函数通常由标签损失、边界框回归损失和IoU损失构成,如下式:
\begin{equation}
    \mathcal{L}(y,\hat{y}) = \lambda_{\text{cls}}\mathcal{L}_{\text{cls}}(c,\hat{c}) +  \lambda_{\text{box}}\mathcal{L}_{\text{box}}(\mathbf{b},\mathbf{\hat{b}}) +  \lambda_{\text{iou}}\mathcal{L}_{\text{iou}}(\mathbf{b},\mathbf{\hat{b}})
  \label{eq:detr-loss}
\end{equation}\par
其中,$\lambda_{\text{cls}}$、$\lambda_{\text{box}}$和$\lambda_{\text{iou}}$为损失权重。与DINO\cite{zhangDINODETRImproved2023}和Deformable DETR\cite{zhuDeformableDETRDeformable2021}中使用的损失函数相同,这里的标签损失$\mathcal{L}_\text{cls}$仍然定义为Focal loss\cite{linFocalLossDense2017}。边界框回归损失定义为Smooth L1 loss。为了增强模型对小尺度陨石坑的检测能力,参照WIoUv3\cite{tongWiseIoUBoundingBox2023}的设计,定义:
\begin{align}
  \mathcal{L}_\text{WIoU}(\mathbf{b},\mathbf{\hat{b}}) &=\frac{\alpha}{\delta\cdot\beta^{\alpha-\delta}}\exp\left[\frac{(x-\hat{x})^2+(y-\hat{y})^2}{(W_g^2+H_g^2)^*}\right]\mathcal{L}_\text{IoU}(\mathbf{b},\mathbf{\hat{b}})\\
  \mathcal{L}_\text{IoU}(\mathbf{b},\mathbf{\hat{b}}) &=1-\frac{\mathbf{b}\cap\mathbf{\hat{b}}}{\mathbf{b}\cup\mathbf{\hat{b}}}\\
  \alpha(\mathbf{b},\mathbf{\hat{b}})&= \frac{\mathcal{L}_\text{IoU}(\mathbf{b},\mathbf{\hat{b}})}{\mathcal{L}^m_\text{IoU}(\mathbf{b},\mathbf{\hat{b}})}
\end{align}\par
式中,右上角的$*$号表示该项在求导运算中视为常量,即从计算图中脱离。$\mathcal{L}^m_\text{IoU}$表示由动量系数$m$控制的指数平滑项$\mathcal{L}_\text{IoU}$,$\beta$和$\delta$是两个超参数,用来抑制低质量的标注框产生的有害梯度。$W_g$和$H_g$分别表示参与计算IoU的预测框与GT框的最小闭包矩形的宽度与高度。$\mathcal{L}_\text{WIoU}$就是计算所得的WIoUv3损失。本文为了进一步增加在小尺度陨石坑上的检测能力,添加与尺度负相关的惩罚项,即尺度越小,相应的损失函数值越大,以增强模型对小尺度陨石坑的响应:
\begin{equation}
  \mathcal{L}_\text{scale}(\mathbf{b},\mathbf{\hat{b}}) = \frac{(\hat{w}-w)^2+(\hat{h}-h)^2}{W_g^2+H_g^2}
\end{equation}\par
由此,最终的小尺度IoU损失函数定义为:
\begin{equation}
  \mathcal{L}_\text{smallIoU}(\mathbf{b},\mathbf{\hat{b}}) = \mathcal{L}_\text{WIoU}(\mathbf{b},\mathbf{\hat{b}})+\mathcal{L}_\text{scale}(\mathbf{b},\mathbf{\hat{b}})
\end{equation}
\subsection{DEM-NMS}
用于训练的数据集中仅为遥感光学图像,不含有地形信息,而检测所得候选陨石坑边界框与真实陨石坑的边界存在差距,因此需要检验才能最终确定陨石坑位置。此外,不同地图块间存在的重叠位置也将产生重复检测结果,应当使用非极大抑制排除重复结果。本文将这两个过程合并,在NMS算法中增加地形信息对检验相同陨石坑的影响,以提高陨石坑检验的准确性,受Softer NMS\cite{heSofterNMSRethinkingBounding2018}和Adaptive NMS\cite{liuAdaptiveNMSRefining2019}的加权融合思路启发,可通过用DEM提供地形数据代替基于学习得到的权重,避免了引入新的回归分支。
\par 根据Liu\cite{liuIdentificationLunarCraters2024}等人描述计算圆形候选框彼此之间的IoU值;根据Wang\cite{wangImprovedGlobalCatalog2021}等人提供的地形修正算法,获取圆形候选框8个径向方向上$\pm20\%$半径范围内的DEM最高点坐标作为边界点(对单个方向径向长度100等分)。DEM-NMS v1版本仅将Liu\cite{liuIdentificationLunarCraters2024}等人提供的圆形IoU和Wang\cite{wangImprovedGlobalCatalog2021}等人提出的地形修正算法简单叠加在一起,在实际使用中,由于小型陨石坑周边最高点往往并不是其真实边缘,以上算法容易产生假边缘。此外,重叠陨石坑在检验时也将由于IoU值过高而被误判为同一陨石坑被抑制, 导致v1版本的DEM-NMS算法产生大量遗漏。这是由于计算IoU未考虑地形高度,而直接从圆形框计算造成的。由于候选陨石坑非常接近真实陨石坑区域,由此可以假设候选陨石坑区域内的最低点所在区域就是真实陨石坑的中心所在区域,v2版本修改IoU计算方式为圆形框内最低点的距离与半径和的比值,即:
\begin{equation}
  \mathrm{CircleIoU}_{12}=\left\{\begin{aligned}
    &0 & \text{if } d_{12}>r_1+r_2\\
    &1-\frac{d_{12}}{r_1+r_2} & \text{otherwise}
  \end{aligned}\right.
  \label{eq:circle-iou}
\end{equation}
\par 式中的$r_1,r_2$分别代表两个圆形框的半径,$d_{12}$代表两个圆形框内\textbf{最低点}的距离。受Softer NMS\cite{heSofterNMSRethinkingBounding2018}启发,待抑制的各个陨石坑参数可经加权后合并,保证多个候选框能够产生更精确的边界。定义各个候选框的权重为:
\begin{equation}
  w_i = \frac{1/d_i^2}{\sum_{j=1}^{N}1/d_j^2}
\end{equation}\par
其中,$d_i$为候选框中心与其最低点的距离。最终的DEM-NMS v2算法流程如下:
\begin{algorithm}[H]
  \caption{DEM-NMS v2版本}
  \label{alg:dem-nms v2}
  \begin{algorithmic}[1]
  \Require $B\in\mathbb{R}^{N\times4}$为候选边界框集合,$N_t$为阈值,$S\in\mathbb{R}^N$为候选边界框得分。\\
  $B=\{b_1,b_2,\ldots,b_n\},S=\{s_1,s_2,\ldots,s_n\}$
  \State $C \gets \{\}$
  \State $T \gets \mathrm{Box2Circle}(B)$
  \State $D \gets \mathrm{getBottom}(T)$ \Comment{获取候选框最低点}
  \State $T=\{t_1,t_2,\ldots,t_n\}, D=\{d_1,d_2,\ldots,d_n\}$
  \While{$T \neq \varnothing $}
    \State $m \gets \mathrm{argmax} S$
    \State $T \gets T - t_m$
    \State $D \gets D - d_m$
    \State $S \gets \mathrm{CircleIoU}(M,T)$ \Comment{计算圆形框IoU}
    \State $S \gets \exp\left(-S^2/\sigma\right)$ \Comment{Soft NMS}
    \State $\mathrm{idx} \gets \mathrm{CircleIoU}(M,B)\ge N_t$ 
    \State $\displaystyle M \gets \frac{\sum_{\mathrm{idx}}b_j/d^2_j}{\sum_{\mathrm{idx}}1/d_j^2}$ \Comment{Softer NMS的加权融合}
    \State $C \gets C\cup M$
  \EndWhile \\
  \Return $C, S$
  \end{algorithmic}
\end{algorithm}
\section{Experiment and Discussion}
\label{sec:experiment and discussion}
\subsection{数据集}
\subsubsection{CraterDANet}
由Yang\cite{yangCraterDANetConvolutionalNeural2022}等人制作,依据NASA的LRO项目发布0.5m/pixel高分辨率的月球NAC图像,覆盖的纬度范围为45°S至46°S、经度范围为176.4°E至178.8°E,共计含有22张800×800分辨率的标注图像对,在标注文件中忽略了直径小于8个像素的陨石坑,标注陨石坑总数接近20000个。
\subsubsection{LRONAC}
由Fairweather\cite{fairweatherAutomaticMappingSmall2022}等人制作,依据NASA的LRO项目发布的16张高分辨率(分辨率从0.25m/pixel至2m/pixel不等)NAC图像经降采样后裁剪至416$\times$416分辨率而得,其中14张来自阿波罗14号着陆点区域,2张来自其他区域。去除直径小于10像素的陨石坑标注,数据集共计含有43402个陨石坑,多数直径小于1km。数据集总共含有248个标注图像对。
\subsubsection{MDCD}
由Yang\cite{yangHighresolutionFeaturePyramid2022}等人制作,含有500张256m/pixel低分辨率火星WAC红外成像图像,覆盖火星全球,即纬度从90°S至90°N、经度范围为180°W至180°E,每张图像裁剪为729×729大小,共计含有12000个陨石坑,单个陨石坑直径大小从6像素至250像素不等。
\subsubsection{小陨石坑数据集}
\label{sec:small crater dataset}由Liu\cite{liuIdentificationLunarCraters2024}、Robbins\cite{robbinsNewGlobalDatabase2019}和Bo\cite{boCatalogueMeterscaleImpact2022}等人提供的陨石坑目录,按照Liu\cite{liuIdentificationLunarCraters2024}提供的圆形NMS去除重叠陨石坑得到融合目录,包含直径5m至500m范围内的中小型陨石坑,覆盖纬度范围为43.090°N至43.022°N,经度范围为52.021°W至52.086°W。将0.5m/pixel高分辨率NAC图像以50\%重叠率滑窗按300×300、500×500大小裁剪,根据球坐标变换和正射投影将目录陨石坑以边界框形式存储为标注文件,去除标注陨石坑个数少于10个的标注文件与对应图像,共计得到74个标注图像对。\par
完整的训练、验证数据集由CraterDANet和LRONAC数据集按比例混合而成,同时为保证检测模型对小尺寸陨石坑的敏感性,将原始数据集中单组标注文件和图像按300$\times$300$\times$300的滑窗和50\%重叠率进一步分割为小图块,去除在图块边缘区域的不完成陨石坑标注数据,保证单幅图像上的陨石坑个数不多于50个。训练集和验证集的比例为9:1,测试集为小陨石坑数据集。
\subsection{训练策略}
本文沿用Co-DETR\cite{zongDETRsCollaborativeHybrid2023}论文中提到表现最佳的DINO-Deformable-DETR模型,在一对一解码器分支上加入噪声以增加解码器的监督训练强度。训练图像均统一缩放至640$\times$640分辨率,使用随机放缩、裁剪和翻转作为数据增强,选取pytorch发布的预训练ResNet50权重作为backbone初始权重。设置2个协作分支,分别对应Faster RCNN\cite{renFasterRCNNRealtime2015}的和ATSS\cite{zhangBridgingGapAnchorbased2020}的分配方法产生协作正样本。初始学习率为$2\times10^{-4}$,优化器选为AdamW,权重衰减系数为$10^{-4}$,学习率衰减系数为0.1,每隔1个epoch衰减一次,训练总共持续12个epoch。训练过程中,每个epoch结束后,使用验证集计算模型的mAP值,选取mAP值最高的模型作为最终模型。在单个NVIDIA RTX 3090和Intel Xeon 4215R CPU@3.20GHz上完成训练和测试,系统内存为376GB。
\subsection{评价指标}
待评价的模型主要包括检测模型和检验算法,其中检测模型和检验算法均在测试集\ref{sec:small crater dataset}上完成测试。
\subsubsection{评价检测模型}
陨石坑检测模型的评价指标一般由不同IoU阈值的精确率和召回率构成。定义正样本为与某个ground truth框的IoU值超过设定阈值的全体检测候选框,其余候选框定义为负样本,通常取正样本IoU阈值为50\%,同时为排除低置信度的检测结果,取定置信分数阈值为0.8。其中精确率(Precision)定义为正确检测的正样本占全体正样本的比例,召回率(Recall)定义为正确检测的正样本占全体ground truth框的比例。为了排除IoU阈值对模型表现的干扰,以0.1步进值计算检测模型在不同阈值下的精确率和召回率之综合表现,即平均精确率(AP)和平均召回率(AR),即有:
\begin{equation}
  \mathrm{P}=\frac{TP}{TP+FP}\qquad\mathrm{R}=\frac{TP}{TP+FN}
  \label{eq:P and R}
\end{equation}
\begin{equation}
  \mathrm{AP}=\int_R\mathrm{P}(R)\mathrm{d}R\qquad\mathrm{AR}=\frac{1}{N}\sum_n\mathrm{R}_n
  \label{eq:AP and AR}
\end{equation}\par
式中,TP表示True Positive,FP表示False Positive,FN表示False Negative。精确率反映模型正确检测陨石坑的能力,召回率反映模型对陨石坑的检测灵敏性。除此以外,本文还关注小目标检测的能力,因此检测评价指标共包括$P$,$R$,$AP_\mathrm{small}$和$AR_\mathrm{small}$。
\subsubsection{评价检验算法}
陨石坑目录是对陨石坑分布、形状的度量,其常用的度量指标包括陨石坑中心点的误差$E_\mathrm{cnt}$、半径误差$E_r$以及IoU$E_\mathrm{IoU}$\cite{liuIdentificationLunarCraters2024}。取定坐标系为月球固定直角坐标系(Moon Fixed),其原点定义为月球中心,x轴沿月球本初子午线与赤道面之交线,z轴垂直于赤道面指向北极,y轴符合右手系,在该坐标系下分别定义以上三个评价指标为:
\begin{align}
  E_\mathrm{cnt}&=\sqrt{\frac{1}{N}\sum_i^N\|\mathbf{x}-\hat{\mathbf{x}}\|_2^2}\\
  E_r&=\sqrt{\frac{1}{N}\sum_i^N\|r_i-\hat{r}_i\|_2^2}\\
  \mathrm{IoU}&=\sqrt{\frac{1}{N}\sum_i^N\mathrm{IoU}_{i\hat{i}}}
\end{align}\par
式中,$\mathbf{x},\hat{\mathbf{x}}$分别代表预测陨石坑的中心坐标和真实陨石坑的中心坐标,$r,\hat{r}$分别代表预测陨石坑和真实陨石坑的半径,单位均为m。最终三个误差值的范围均在0~1以内,误差数值越小,检验算法效果越好。
\subsection{目标检测实验}
\subsubsection{与SOTA模型的对比}

本文与当前主流的陨石坑检测框架如Faster RCNN、YOLOv7和SSD进行对比,与目标检测主流框架DINO、Co-DINO和RT-DETR进行对比。此外,为了验证本文提出的小目标损失函数的有效性,还与小目标检测领域的DNTR等论文进行对比。
\par 此外,由于建立陨石坑目录的区域尚未存在为标注文件,应当考察检测模型在零样本条件下的泛化性能,建立零样本测试基准集\cite{boCatalogueMeterscaleImpact2022}上计算其指标$\mathrm{P,R,AP_small}$和$\mathrm{AR_small}$。此外,还与当前目标检测领域的先进模型进行对比,所有模型的backbone均取为ResNet50。训练的迭代次数与评价指标一并列出。
\begin{table}[H]
  \begin{center}
  \caption{在Bo\cite{boCatalogueMeterscaleImpact2022}数据集上的对比实验}
  \label{tab:detect-comp}
  \begin{tabular}{ c  c  c  c  c  c c}
  \toprule
  模型 & 训练迭代数 & 
  Precision(\%)$\uparrow$ & Recall(\%)$\uparrow$ & $\mathrm{AP}_\mathrm{small}$(\%)$\uparrow$ & $\mathrm{AR}_\mathrm{small}$(\%)$\uparrow$& $\mathrm{mAP}\uparrow$ \\
  \hline
  基于CNN & & & & & & \\
  Faster-RCNN & 30 & 84.58 & 49.83 & 23.5 & 32.6 & 24.1\\
  YOLOv7 & 74 & 39.7 & 40.6 & -- & -- & 21.6 \\
  YOLOv11 & 120 & 81.02 & 19.68 & -- & -- & 31.78\\
  DiffusionDet &120 & 98.4 & 82.74 & 84.4 & 69.9&61.9  \\
  SSD & 30 & 91.3 &33.5 &22.6 &35.2&22.1 \\
  \hline
  基于Transformer & & & & && \\
  RT-DETR & 36 & 93.0 & 56.16 & 35.0 & 40.9& 34.4\\
  Align DETR&30 & 100 & 53.77 & 38.80 & 54.70 & 33.80\\
  DINO &30 &98.65 &72.61 &42.60 &57.10 &37.10\\
  Co-DINO &30 &98.67 & 93.04 & 54.90 & 65.60 & 51.11\\
  Co-DINO$\ddagger$ (ours)& & & & & &\\
  \hline
  小目标检测 & & & & & & \\
  DNTR\cite{liuDeNoisingFPNTransformer2024}& & & & & &  \\
  \bottomrule 
  \end{tabular}
  \end{center}
\end{table}
说明,带有$\ddagger$的模型为本文修改小目标损失后的改进版本。
\par 或许可以在这里加入一个在MDCD数据集上的对比实验(不一定需要零样本测试)

\subsubsection{协作分支的作用}

\subsubsection{小尺度损失的作用}

\subsection{检验算法实验}
检测所得陨石坑标注,经过坐标系平移变换,可统一至相同的图像坐标系下。此时,重叠区域的陨石坑检测结果应当通过检测算法去除,并与地形信息结合验证,可得到图像坐标系下的陨石坑目录,通过投影逆变换后,可得到月球坐标系下的陨石坑目录。
\subsubsection{与其他检验算法对比}
在Bo\cite{boCatalogueMeterscaleImpact2022}数据集上,将本文提出的soft NMS+地形结合检验算法与采用普通NMS算法、普通NMS+地形结合算法作比较,计算其误差值$E_\mathrm{cnt},E_r,E_\mathrm{IoU}$,所得结果一并如下表列出:
\begin{table}[H]
  \begin{center}
  \caption{在Bo\cite{boCatalogueMeterscaleImpact2022}数据集上的对比实验}
  \label{tab:verify-comp}
  \begin{tabular}{ c  c  c  c }
  \toprule
  算法 & $E_\mathrm{cnt}$ & $E_r$ & $E_\mathrm{IoU}$ \\
  \hline
  普通NMS &  & 0.08 & 0.05 \\
  普通NMS+地形结合 & 0.09 & 0.06 & 0.03 \\
  soft NMS & 0.07 & 0.05 & 0.02 \\
  soft NMS+地形结合 & 0.06 & 0.04 & 0.01 \\
  \bottomrule 
  \end{tabular}
  \end{center}
\end{table}\par
可以看到,DEM-NMS算法处理后得到的陨石坑目录在中心点、半径和IoU误差上均优于其他算法。
\subsubsection{讨论}
\subsection{构建陨石坑目录}
\subsubsection{嫦娥五号着陆点区域}
\label{sec:chang_e_5 crater}
用于构建陨石坑目录的数据包括遥感光学图像与遥感DEM,首先需要配准。这里使用的遥感光学图像来自于LROC\cite{robinsonLunarReconnaissanceOrbiter2010}项目提供的NAC\_ROI\_CHANGE5\_LOA\_E432N3091.IMG\footnote{ 下载链接见于https://pds.lroc.asu.edu/data/LRO-L-LROC-5-RDR-V1.0/LROLRC\_2001/DATA/BDR/NAC\_ROI/CHANGE5\_LOA。},使用的遥感DEM数据来自于Qian\cite{qianCopernicanaged200Ma2021}等人提供的CE5\_NAC DEM\footnote{下载链接见于https://zenodo.org/records/5532994。}和LROC项目提供的NAC\_DTM\_CHANGE501.tif\footnote{ 下载链接见于https://wms.lroc.asu.edu/lroc/view\_rdr\_product/NAC\_DTM\_CHANGE501},将两个地形数据合并后与光学图像配准,采用相同的等距圆柱投影方式。由于等距圆柱投影在远离标准纬线位置处产生较大失真,根据嫦娥五号着陆点位置和NAC\_ROI\_CHANGE5\_LOA\_E432N3091.IMG文件提供的投影方式,将等距圆柱投影标准纬线移动至43.8°N处,将中央经线移动到51°W处,使用最近邻内插将DEM与遥感光学图像像素对齐,制成两幅分辨率为1.2m/pixel的区域地图和区域DEM,总计分辨率达到5298×7515。为检测更多的小尺度陨石坑,在制作推理数据集时,选取300×300、500×500和1000×1000三种图像分辨率层次,按50\%重叠率的滑动窗分割区域地图和区域DEM制成推理图像集。\par
经过DEM-NMS运算后,为了保证较高的召回率,设定其置信分数阈值为0.3,去除得分少于阈值的陨石坑后,总共得到XXXX个陨石坑,列出不同尺度分布下的陨石坑密度和分布情况:
\begin{table}[H]
  \begin{center}
  \caption{嫦娥五号着陆点区域陨石坑尺度及分布密度}
  \label{tab:chang_e_5 crater}
  \begin{tabular}{ c  c  c  c}
  \toprule
  直径范围 & 数目 & 密度 & 平均深度 \\
  \hline
  5m-10m &  &  & \\
  10-50m &  &  & \\
  50-100m &  &  & \\
  >100m &  &  & \\
  \bottomrule 
  \end{tabular}
  \end{center}
\end{table}\par
在当前地图下绘制陨石坑的分布图如图所示:

% \begin{figure}[H]
%   \centering
%   \includegraphics[width=0.8\textwidth]{figures/chang_e_5_crater.png}
%   \caption{嫦娥五号着陆点区域陨石坑分布图}
%   \label{fig:chang_e_5 crater}
% \end{figure}

\subsection{月球南极区域}
采用与\ref{sec:chang_e_5 crater}节相似的方法构建,这里使用来自于LROC项目提供的月球南极1m/pixel高分辨率极区投影图像\footnote{ 下载链接见于https://pds.lroc.asu.edu/data/LRO-L-LROC-5-RDR-V1.0/LROLRC\_2001/DATA/BDR/NAC\_POLE/NAC\_POLE\_SOUTH。},由于月球南极地区太阳高度角极低,同一张遥感图像内含有较多阴影暗区,LROC采取将多组不同时间戳取得的遥感光学图像拼接,;使用的遥感DEM数据来自于NASA的LOLA\cite{barkerImprovedLOLAElevation2021}项目提供的LDEM\_875S\_5M\_FLOAT.IMG\footnote{下载链接见于https://imbrium.mit.edu/DATA/LOLA\_GDR/POLAR/FLOAT\_IMG。},分辨率为5m/pixel。LDEM包含的数据为靠近南极点的87.5$^\circ$S~90$^\circ$S纬度范围,对应的光学图像为NAC\_POLE\_P870S0150.IMG起至NAC\_POLE\_P892S3150.IMG终止共计24张影像,均采用极点位于月球南极点的极区投影方式。南极地区光照较为复杂,LROC将多组来自不同时间戳处获取的遥感光学图像进行拼接,以获取光照相对均匀的拼图。需要注意的是,南极地区存在永久阴影区,这些区域内的陨石坑无法通过光学图像检测,因此本文在构建月球南极地区的陨石坑目录时,将永久阴影区排除在外。

\par 由于DEM数据与遥感光学图像采用的是相同的极区投影方式,由此这里不需要再作配准计算。\par
生的阈值和经坐标系转换、投影逆运算后,将检验确认得到的全体陨石坑转换至月球坐标系下
\section{Conclusion}
\label{sec:conclusion}
\input{chapters/acknowledgement.tex}
\newpage
\printbibliography[heading=bibliography,title=参考文献]
\end{document}