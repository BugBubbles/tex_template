\section{Method}
\label{sec:method}
构建陨石坑目录需要陨石坑检测和检验两大步骤,本章节着重介绍用于目标检测的CO-DINO网络构造原理、训练策略以及用于陨石坑检验的DEM-NMS算法。
\subsection{协作分支}
极度不平衡的正负查询比例,使得基于DETR框架检测模型在训练过程中,有效的一一对应匹配梯度难以回传至深层网络,即Encoder上,导致Encoder层内的注意力模块难以被正确监督学习,降低了其检测性能。Co-DETR\cite{zongDETRsCollaborativeHybrid2023}通过引入大量的协作分支,在单个协作分支上使用随机选定的边界框分配方式,为每一个ground truth边界框分配多个协作框,形成类似于锚点机制的边界框分配方式,极大丰富了正查询样本的占比,从而保证正负查询比例协调。Co-DETR仅使用了最浅层的特征图,即最粗糙的特征图,经过多次降采样和卷积运算后,按照simple feature pyramids\cite{liExploringPlainVision2022}的构成方法产生特征金字塔,作为其协作分支的输入。协作分支各自的损失函数定义为:
\begin{equation}
  \mathcal{L}_\text{co}(y,\hat{y}) = \mathcal{L}_i(c^\text{pos}_i,\hat{c}_i^{pos})+\mathcal{L}_i(c^\text{neg}_i, \hat{c}_i^\text{neg})
\end{equation}\par
其中,$y=(c,\mathbf{b})=(c,(x,y,w,h))\in\mathbb{R}\times\mathbb{R}^{4}$为ground truth,包括了标签值与边界框参数,$\hat{y}$为对应预测值,$c$为GT类别,$\hat{c}$为预测类别,$\mathbf{b}$为GT边界框,$\mathbf{\hat{b}}$为预测边界框,上标pos和neg分别表示第i个分配方式决定下的正负样本。协作分支的总损失函数定义为:
\begin{equation}
  \mathcal{L}_\text{co} = \sum_{i=1}^{N}\mathcal{L}_i(y,\hat{y})
  \label{eq:co-loss}
\end{equation}\par
其中,$N$为协作分支的数量。实际上,不仅是分配的正样本标签,正样本的边界框坐标回归也应当参与至协作分支的损失函数中,以进一步增强协作分支监督学习对Encoder的作用。这里选用简单的Smooth L1 loss作为边界框回归损失函数,将式\ref{eq:co-loss}改写为:
\begin{equation}
  \mathcal{L}_\text{co} = \sum_{i=1}^{N}\left[\mathcal{L}_i(c^\text{pos}_i,\hat{c}_i^{pos})+\mathcal{L}_i(c^\text{neg}_i, \hat{c}_i^\text{neg})+\lambda\mathcal{L}_{\text{box},i}(\mathbf{b}^\text{pos}_i,\mathbf{\hat{b}}_i^\text{pos})\right]
  \label{eq:co-loss-reg}
\end{equation}\par
\subsection{小尺度损失函数}
基于DETR框架的检测模型,其训练的损失函数通常由标签损失、边界框回归损失和IoU损失构成,如下式:
\begin{equation}
    \mathcal{L}(y,\hat{y}) = \lambda_{\text{cls}}\mathcal{L}_{\text{cls}}(c,\hat{c}) +  \lambda_{\text{box}}\mathcal{L}_{\text{box}}(\mathbf{b},\mathbf{\hat{b}}) +  \lambda_{\text{iou}}\mathcal{L}_{\text{iou}}(\mathbf{b},\mathbf{\hat{b}})
  \label{eq:detr-loss}
\end{equation}\par
其中,$\lambda_{\text{cls}}$、$\lambda_{\text{box}}$和$\lambda_{\text{iou}}$为损失权重。与DINO\cite{zhangDINODETRImproved2023}和Deformable DETR\cite{zhuDeformableDETRDeformable2021}中使用的损失函数相同,这里的标签损失$\mathcal{L}_\text{cls}$仍然定义为Focal loss\cite{linFocalLossDense2017}。边界框回归损失定义为Smooth L1 loss。为了增强模型对小尺度陨石坑的检测能力,参照WIoUv3\cite{tongWiseIoUBoundingBox2023}的设计,定义:
\begin{align}
  \mathcal{L}_\text{WIoU}(\mathbf{b},\mathbf{\hat{b}}) &=\frac{\alpha}{\delta\cdot\beta^{\alpha-\delta}}\exp\left[\frac{(x-\hat{x})^2+(y-\hat{y})^2}{(W_g^2+H_g^2)^*}\right]\mathcal{L}_\text{IoU}(\mathbf{b},\mathbf{\hat{b}})\\
  \mathcal{L}_\text{IoU}(\mathbf{b},\mathbf{\hat{b}}) &=1-\frac{\mathbf{b}\cap\mathbf{\hat{b}}}{\mathbf{b}\cup\mathbf{\hat{b}}}\\
  \alpha(\mathbf{b},\mathbf{\hat{b}})&= \frac{\mathcal{L}_\text{IoU}(\mathbf{b},\mathbf{\hat{b}})}{\mathcal{L}^m_\text{IoU}(\mathbf{b},\mathbf{\hat{b}})}
\end{align}\par
式中,右上角的$*$号表示该项在求导运算中视为常量,即从计算图中脱离。$\mathcal{L}^m_\text{IoU}$表示由动量系数$m$控制的指数平滑项$\mathcal{L}_\text{IoU}$,$\beta$和$\delta$是两个超参数,用来抑制低质量的标注框产生的有害梯度。$W_g$和$H_g$分别表示参与计算IoU的预测框与GT框的最小闭包矩形的宽度与高度。$\mathcal{L}_\text{WIoU}$就是计算所得的WIoUv3损失。本文为了进一步增加在小尺度陨石坑上的检测能力,添加与尺度负相关的惩罚项,即尺度越小,相应的损失函数值越大,以增强模型对小尺度陨石坑的响应:
\begin{equation}
  \mathcal{L}_\text{scale}(\mathbf{b},\mathbf{\hat{b}}) = \frac{(\hat{w}-w)^2+(\hat{h}-h)^2}{W_g^2+H_g^2}
\end{equation}\par
由此,最终的小尺度IoU损失函数定义为:
\begin{equation}
  \mathcal{L}_\text{smallIoU}(\mathbf{b},\mathbf{\hat{b}}) = \mathcal{L}_\text{WIoU}(\mathbf{b},\mathbf{\hat{b}})+\mathcal{L}_\text{scale}(\mathbf{b},\mathbf{\hat{b}})
\end{equation}
\subsection{DEM-NMS}
用于训练的数据集中仅为遥感光学图像,不含有地形信息,而检测所得候选陨石坑边界框与真实陨石坑的边界存在差距,因此需要检验才能最终确定陨石坑位置。此外,不同地图块间存在的重叠位置也将产生重复检测结果,应当使用非极大抑制排除重复结果。本文将这两个过程合并,在NMS算法中增加地形信息对检验相同陨石坑的影响,以提高陨石坑检验的准确性,受Softer NMS\cite{heSofterNMSRethinkingBounding2018}和Adaptive NMS\cite{liuAdaptiveNMSRefining2019}的加权融合思路启发,可通过用DEM提供地形数据代替基于学习得到的权重,避免了引入新的回归分支。
\par 根据Liu\cite{liuIdentificationLunarCraters2024}等人描述计算圆形候选框彼此之间的IoU值;根据Wang\cite{wangImprovedGlobalCatalog2021}等人提供的地形修正算法,获取圆形候选框8个径向方向上$\pm20\%$半径范围内的DEM最高点坐标作为边界点(对单个方向径向长度100等分)。DEM-NMS v1版本仅将Liu\cite{liuIdentificationLunarCraters2024}等人提供的圆形IoU和Wang\cite{wangImprovedGlobalCatalog2021}等人提出的地形修正算法简单叠加在一起,在实际使用中,由于小型陨石坑周边最高点往往并不是其真实边缘,以上算法容易产生假边缘。此外,重叠陨石坑在检验时也将由于IoU值过高而被误判为同一陨石坑被抑制, 导致v1版本的DEM-NMS算法产生大量遗漏。这是由于计算IoU未考虑地形高度,而直接从圆形框计算造成的。由于候选陨石坑非常接近真实陨石坑区域,由此可以假设候选陨石坑区域内的最低点所在区域就是真实陨石坑的中心所在区域,v2版本修改IoU计算方式为圆形框内最低点的距离与半径和的比值,即:
\begin{equation}
  \mathrm{CircleIoU}_{12}=\left\{\begin{aligned}
    &0 & \text{if } d_{12}>r_1+r_2\\
    &1-\frac{d_{12}}{r_1+r_2} & \text{otherwise}
  \end{aligned}\right.
  \label{eq:circle-iou}
\end{equation}
\par 式中的$r_1,r_2$分别代表两个圆形框的半径,$d_{12}$代表两个圆形框内\textbf{最低点}的距离。受Softer NMS\cite{heSofterNMSRethinkingBounding2018}启发,待抑制的各个陨石坑参数可经加权后合并,保证多个候选框能够产生更精确的边界。定义各个候选框的权重为:
\begin{equation}
  w_i = \frac{1/d_i^2}{\sum_{j=1}^{N}1/d_j^2}
\end{equation}\par
其中,$d_i$为候选框中心与其最低点的距离。最终的DEM-NMS v2算法流程如下:
\begin{algorithm}[H]
  \caption{DEM-NMS v2版本}
  \label{alg:dem-nms v2}
  \begin{algorithmic}[1]
  \Require $B\in\mathbb{R}^{N\times4}$为候选边界框集合,$N_t$为阈值,$S\in\mathbb{R}^N$为候选边界框得分。\\
  $B=\{b_1,b_2,\ldots,b_n\},S=\{s_1,s_2,\ldots,s_n\}$
  \State $C \gets \{\}$
  \State $T \gets \mathrm{Box2Circle}(B)$
  \State $D \gets \mathrm{getBottom}(T)$ \Comment{获取候选框最低点}
  \State $T=\{t_1,t_2,\ldots,t_n\}, D=\{d_1,d_2,\ldots,d_n\}$
  \While{$T \neq \varnothing $}
    \State $m \gets \mathrm{argmax} S$
    \State $T \gets T - t_m$
    \State $D \gets D - d_m$
    \State $S \gets \mathrm{CircleIoU}(M,T)$ \Comment{计算圆形框IoU}
    \State $S \gets \exp\left(-S^2/\sigma\right)$ \Comment{Soft NMS}
    \State $\mathrm{idx} \gets \mathrm{CircleIoU}(M,B)\ge N_t$ 
    \State $\displaystyle M \gets \frac{\sum_{\mathrm{idx}}b_j/d^2_j}{\sum_{\mathrm{idx}}1/d_j^2}$ \Comment{Softer NMS的加权融合}
    \State $C \gets C\cup M$
  \EndWhile \\
  \Return $C, S$
  \end{algorithmic}
\end{algorithm}