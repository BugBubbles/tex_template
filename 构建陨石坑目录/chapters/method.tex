\section{Method}
\label{sec:method}
构建陨石坑目录需要陨石坑检测和检验两大步骤,本章节着重介绍用于目标检测的CO-DINO网络构造原理、训练策略以及用于陨石坑检验的DEM-NMS算法。
\subsection{预处理}
用于构建陨石坑目录的数据包括遥感光学图像与遥感DEM,首先需要配准。这里使用的遥感光学图像来自于LROC项目提供的NAC\_ROI\_CHANGE5\_LOA\_E432N3091.IMG\footnote{ 下载链接见于https://pds.lroc.asu.edu/data/LRO-L-LROC-5-RDR-V1.0/LROLRC\_2001/DATA/BDR/NAC\_ROI/CHANGE5\_LOA。},使用的遥感DEM数据来自于Qian\cite{qianCopernicanaged200Ma2021}等人提供的CE5\_NAC DEM\footnote{下载链接见于https://zenodo.org/records/5532994。}和LROC项目提供的NAC\_DTM\_CHANGE501.tif\footnote{ 下载链接见于https://wms.lroc.asu.edu/lroc/view\_rdr\_product/NAC\_DTM\_CHANGE501},将两个地形数据合并后与光学图像配准,采用相同的等距圆柱投影方式。由于等距圆柱投影在远离标准纬线位置处产生较大失真,根据嫦娥五号着陆点位置和NAC\_ROI\_CHANGE5\_LOA\_E432N3091.IMG文件提供的投影方式,将等距圆柱投影标准纬线移动至43.8°N处,将中央经线移动到51°W处,使用最近邻内插将DEM与遥感光学图像像素对齐,制成两幅分辨率为1.2m/pixel的区域地图和区域DEM,总计分辨率达到5298×7515。为检测更多的小尺度陨石坑,在制作推理数据集时,选取300×300、500×500和1000×1000三种图像分辨率层次,按50\%重叠率的滑动窗分割区域地图和区域DEM制成推理图像集。
\subsection{协作分支}
\subsection{去噪训练}
\subsection{小尺度损失函数}
\subsection{DEM-NMS}
