\section{Method}
\label{sec:method}
构建陨石坑目录需要陨石坑检测和检验两大步骤,本章节着重介绍用于目标检测的CO-DINO网络构造原理、训练策略以及用于陨石坑检验的DEM-NMS算法。
\subsection{预处理}
用于构建陨石坑目录的数据包括遥感光学图像与遥感DEM,首先需要配准。这里使用的遥感光学图像来自于LROC项目提供的NAC\_ROI\_CHANGE5\_LOA\_E432N3091.IMG\footnote{ 下载链接见于https://pds.lroc.asu.edu/data/LRO-L-LROC-5-RDR-V1.0/LROLRC\_2001/DATA/BDR/NAC\_ROI/CHANGE5\_LOA。},使用的遥感DEM数据来自于Qian\cite{qianCopernicanaged200Ma2021}等人提供的CE5\_NAC DEM\footnote{下载链接见于https://zenodo.org/records/5532994。}和LROC项目提供的NAC\_DTM\_CHANGE501.tif\footnote{ 下载链接见于https://wms.lroc.asu.edu/lroc/view\_rdr\_product/NAC\_DTM\_CHANGE501},将两个地形数据合并后与光学图像配准,采用相同的等距圆柱投影方式。由于等距圆柱投影在远离标准纬线位置处产生较大失真,根据嫦娥五号着陆点位置和NAC\_ROI\_CHANGE5\_LOA\_E432N3091.IMG文件提供的投影方式,将等距圆柱投影标准纬线移动至43.8°N处,将中央经线移动到51°W处,使用最近邻内插将DEM与遥感光学图像像素对齐,制成两幅分辨率为1.2m/pixel的区域地图和区域DEM,总计分辨率达到5298×7515。为检测更多的小尺度陨石坑,在制作推理数据集时,选取300×300、500×500和1000×1000三种图像分辨率层次,按50\%重叠率的滑动窗分割区域地图和区域DEM制成推理图像集。
\subsection{协作分支}
极度不平衡的正负查询比例,使得基于DETR框架检测模型在训练过程中,有效的一一对应匹配梯度难以回传至深层网络,即Encoder上,导致Encoder层内的注意力模块难以被正确监督学习,降低了其检测性能。Co-DETR\cite{zongDETRsCollaborativeHybrid2023}通过引入大量的协作分支,在单个协作分支上使用随机选定的边界框分配方式,为每一个ground truth边界框分配多个协作框,形成类似于锚点机制的边界框分配方式,极大丰富了正查询样本的占比,从而保证正负查询比例协调。Co-DETR仅使用了最浅层的特征图,即最粗糙的特征图,经过多次降采样和卷积运算后,按照simple feature pyramids\cite{liExploringPlainVision2022}的构成方法产生特征金字塔,作为其协作分支的输入。协作分支各自的损失函数定义为:
\begin{equation}
  \mathcal{L}_\text{co}(y,\hat{y}) = \mathcal{L}_i(c^\text{pos}_i,\hat{c}_i^{pos})+\mathcal{L}_i(c^\text{neg}_i, \hat{c}_i^\text{neg})
\end{equation}\par
其中,$y=(c,\mathbf{b})=(c,(x,y,w,h))\in\mathbb{R}\times\mathbb{R}^{4}$为ground truth,包括了标签值与边界框参数,$\hat{y}$为对应预测值,$c$为GT类别,$\hat{c}$为预测类别,$\mathbf{b}$为GT边界框,$\mathbf{\hat{b}}$为预测边界框,上标pos和neg分别表示第i个分配方式决定下的正负样本。协作分支的总损失函数定义为:
\begin{equation}
  \mathcal{L}_\text{co} = \sum_{i=1}^{N}\mathcal{L}_i(y,\hat{y})
  \label{eq:co-loss}
\end{equation}\par
其中,$N$为协作分支的数量。实际上,不仅是分配的正样本标签,正样本的边界框坐标回归也应当参与至协作分支的损失函数中,以进一步增强协作分支监督学习对Encoder的作用。这里选用简单的Smooth L1 loss作为边界框回归损失函数,将式\ref{eq:co-loss}改写为:
\begin{equation}
  \mathcal{L}_\text{co} = \sum_{i=1}^{N}\left[\mathcal{L}_i(c^\text{pos}_i,\hat{c}_i^{pos})+\mathcal{L}_i(c^\text{neg}_i, \hat{c}_i^\text{neg})+\lambda\mathcal{L}_{\text{box},i}(\mathbf{b}^\text{pos}_i,\mathbf{\hat{b}}_i^\text{pos})\right]
  \label{eq:co-loss-reg}
\end{equation}\par
\subsection{小尺度损失函数}
基于DETR框架的检测模型,其训练的损失函数通常由标签损失、边界框回归损失和IoU损失构成,如下式:
\begin{equation}
    \mathcal{L}(y,\hat{y}) = \lambda_{\text{cls}}\mathcal{L}_{\text{cls}}(c,\hat{c}) +  \lambda_{\text{box}}\mathcal{L}_{\text{box}}(\mathbf{b},\mathbf{\hat{b}}) +  \lambda_{\text{iou}}\mathcal{L}_{\text{iou}}(\mathbf{b},\mathbf{\hat{b}})
  \label{eq:detr-loss}
\end{equation}\par
其中,$\lambda_{\text{cls}}$、$\lambda_{\text{box}}$和$\lambda_{\text{iou}}$为损失权重。与DINO\cite{zhangDINODETRImproved2023}和Deformable DETR\cite{zhuDeformableDETRDeformable2021}中使用的损失函数相同,这里的标签损失$\mathcal{L}_\text{cls}$仍然定义为Focal loss\cite{linFocalLossDense2017}。边界框回归损失定义为Smooth L1 loss。为了增强模型对小尺度陨石坑的检测能力,参照WIoUv3\cite{tongWiseIoUBoundingBox2023}的设计,定义:
\begin{align}
  \mathcal{L}_\text{WIoU}(\mathbf{b},\mathbf{\hat{b}}) &=\frac{\alpha}{\delta\cdot\beta^{\alpha-\delta}}\exp\left[\frac{(x-\hat{x})^2+(y-\hat{y})^2}{(W_g^2+H_g^2)^*}\right]\mathcal{L}_\text{IoU}(\mathbf{b},\mathbf{\hat{b}})\\
  \mathcal{L}_\text{IoU}(\mathbf{b},\mathbf{\hat{b}}) &=1-\frac{\mathbf{b}\cap\mathbf{\hat{b}}}{\mathbf{b}\cup\mathbf{\hat{b}}}\\
  \alpha(\mathbf{b},\mathbf{\hat{b}})&= \frac{\mathcal{L}_\text{IoU}(\mathbf{b},\mathbf{\hat{b}})}{\mathcal{L}^m_\text{IoU}(\mathbf{b},\mathbf{\hat{b}})}
\end{align}\par
式中,右上角的$*$号表示该项在求导运算中视为常量,即从计算图中脱离。$\mathcal{L}^m_\text{IoU}$表示由动量系数$m$控制的指数平滑项$\mathcal{L}_\text{IoU}$,$\beta$和$\delta$是两个超参数,用来抑制低质量的标注框产生的有害梯度。$W_g$和$H_g$分别表示参与计算IoU的预测框与GT框的最小闭包矩形的宽度与高度。$\mathcal{L}_\text{WIoU}$就是计算所得的WIoUv3损失。本文为了进一步增加在小尺度陨石坑上的检测能力,添加与尺度负相关的惩罚项,即尺度越小,相应的损失函数值越大,以增强模型对小尺度陨石坑的响应:
\begin{equation}
  \mathcal{L}_\text{scale}(\mathbf{b},\mathbf{\hat{b}}) = \frac{(x-\hat{x})^2+(y-\hat{y})^2}{\hat{w}^2+\hat{h}^2}+\frac{(\hat{w}-w)^2+(\hat{h}-h)^2}{\hat{w}^2+\hat{h}^2}
\end{equation}\par
由此,最终的小尺度IoU损失函数定义为:
\begin{equation}
  \mathcal{L}_\text{smallIoU}(\mathbf{b},\mathbf{\hat{b}}) = \mathcal{L}_\text{WIoU}(\mathbf{b},\mathbf{\hat{b}})+\mathcal{L}_\text{scale}(\mathbf{b},\mathbf{\hat{b}})
\end{equation}
\subsection{DEM-NMS}
用于训练的数据集中仅为遥感光学图像,不含有地形信息,而检测所得候选陨石坑边界框与真实陨石坑的边界存在差距,因此需要加以检验才能最终确定陨石坑位置。此外,不同地图块间存在的重叠位置也将产生重复检测结果,应当使用非极大抑制排除重复结果。本文将这两个过程合并,在NMS算法中增加地形信息对检验相同陨石坑的影响,以提高陨石坑检验的准确性,受Softer NMS\cite{heSofterNMSRethinkingBounding2018}和Adaptive NMS\cite{liuAdaptiveNMSRefining2019}的加权融合思路启发,可通过用DEM提供地形数据代替基于学习得到的权重,避免了引入新的回归分支。

DEM-NMS算法的流程如下:\par

\begin{algorithm}[H]
  \caption{DEM-NMS}
  \KwIn{检测模型输出的候选陨石坑边界框集合$\mathcal{B}$,对应的置信度集合$\mathcal{C}$,地形DEM}
  \KwOut{最终确定的陨石坑边界框集合$\mathcal{B}'$}
  \For{$\mathbf{b}_i\in\mathcal{B}$}{
    计算$\mathbf{b}_i$的中心点坐标$(x_i,y_i)$\;
    计算$\mathbf{b}_i$的面积$S_i$\;
    计算$\mathbf{b}_i$的高度$h_i$和宽度$w_i$\;
    计算$\mathbf{b}_i$的置信度$c_i$\;
    计算$\mathbf{b}_i$的高度和宽度的比值$r_i=\frac{h_i}{w_i}$\;
    计算$\mathbf{b}_i$的面积与置信度的乘积$S_i\times c_i$\;
    计算$\mathbf{b}_i$的高度和宽度的比值与置信度的乘积$r_i\times c_i$\;
  }
  对$\mathcal{B}$按照$S_i\times c_i$降序排列,得到$\mathcal{B}_S$\;
  对$\mathcal{B}$按照$r_i\times c_i$降序排列,得到$\mathcal{B}_R$\;
  \For{$\mathbf{b}_i\in\mathcal{B}_S$}{
    \If{$\mathbf{b}_i$未被处理}{
      \For{$\mathbf{b}_j\in\mathcal{B}_R$}{
        \If{$\mathbf{b}_j$未被处理}{
          计算$\mathbf{b}_i$与$\mathbf{b}_j$的IoU\;
          \If{$\mathbf{b}_i$与$\mathbf{b}_j$的IoU大于阈值}{
            计算$\mathbf{b}_i$与$\mathbf{b}_j$的DEM差值\;
            \If{$\mathbf{b}_i$的DEM差值小于$\mathbf{b}_j$的DEM差值}{
              标记$\mathbf{b}_j$为已处理\;
            }
            \Else{
              标记$\mathbf{b}_i$为已处理\;
            }
          }
        }
      }
    }
  }
  \For{$\mathbf{b}_i\in\mathcal{B}$}{
    \If{$\mathbf{b}_i$未被处理}{
      将$\mathbf{b}_i$加入$\mathcal{B}'$\;
    }
  }
\end{algorithm}
