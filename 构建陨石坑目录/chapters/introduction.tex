\section{Introduction}
深空探测是当前航天领域的研究热点,自主导航能力是深空探测航天器的必要能力之一。地外天体表面丰富的陨石坑能够作为有效路标,可用于航天器绕飞、着陆等过程的自主导航,因此,获取高精度、详尽的地外天体陨石坑目录,尤其是中小型(10-100m)陨石坑目录,是进行深空探测后续任务的基础。
嫦娥五号于2020年12月1日成功着陆于月球虹湾地区43.06°N,51.92°W,围绕其着陆点,中国航天局和NASA等机构产出大量深空遥感图像,便于构建嫦娥五号着陆点附近的陨石坑目录。深空遥感图像类型包括光学遥感图像、红外遥感图像、数字高程地图以及数字地形地图等,通过遥感图像作光度校正、几何校正等预处理手段,可得到正射的遥感图像,此时陨石坑形式均呈现为规则的圆形外形,便于通过陨石坑检测方法提取其中心和半径,提取陨石坑的算法被称为“陨石坑检测算法CDA”。
根据提取陨石坑的技术路径,可将陨石坑检测方法分为基于手工设计的特征和基于数据驱动的特征两种类型。基于手工设计的特征包括Hough变换特征\cite{emamiCraterDetectionUsing2019}、阴影-高亮特征\cite{maassEdgefreeScalePose2011,maassRobustApproximationImage2016}、Canny边缘\cite{emamiCraterDetectionUsing2019}、地形路径特征\cite{wangNovelApproachMultiscale2022}、小波模板\cite{diMachineLearningApproach2014,wangActiveMachineLearning2019}和匹配模板\cite{bandeiraDetectionSubkilometerCraters2012}等,通过对特定区域或特征遥感图像的统计分析,手工设计的特征可取得较高的陨石坑检测召回率与精确率,但是这类方法必须针对不同区域、不同遥感图像修改模型参数,在处理较多图像数据时很不方便,且手工设计的指标往往还需要针对不同特征设计不同的检测后端算法,特征的使用不具有通用性。深度学习和计算机视觉的发展改变了传统的特征设计方式,基于数据驱动的特征能够在数据集上自主学习、更新,并保证在训练集上取得最佳性能,是当前设计自动化陨石坑检测所采用的主流特征。根据深度学习采用的特征提取器类型,可将特征分为基于VGG和ResNet\cite{wangCraterIDNetEndendFully2018,yangLunarImpactCrater2020,chenCraterDetectionRecognition2021}、基于UNet\cite{silburtLunarCraterIdentification2019,songEffectiveLunarCrater2020}、基于Transformer\cite{daiBoostingCraterDetection2023}、基于SNN\cite{zhaEnergyefficientCratersDetection2024}等,其输出的高维向量特征较为复杂,难以可视化,往往通过目标检测头网络完成陨石坑中心坐标与宽高范围的边界框回归计算,由此得到完整的深度学习陨石坑检测网络,如基于单阶段模型的YOLOv3\cite{fairweatherAutomaticMappingSmall2022}、YOLOv5\cite{liuTwostageAdaptiveNetwork2024}、YOLOv7\cite{daiBoostingCraterDetection2023}、SSD\cite{silvestriniOpticalNavigationLunar2022};基于两阶段模型的Faster RCNN\cite{yangLunarImpactCrater2020,yangCraterDANetConvolutionalNeural2022,liuIdentificationLunarCraters2024}、CraterIDNet\cite{wangCraterIDNetEndendFully2018}和HRFPNet\cite{yangHighresolutionFeaturePyramid2022}、LCDNet\cite{miaoLCDNetInnovativeNeural2024};基于无锚点模型的CenterNet\cite{zhangAutomaticDetectionSmallscale2024};基于端到端检测模型的Crater-DETR\cite{guoCraterDETRNovelTransformer2024}。这些目标检测网络能够在几何校正、光度校正后的正射遥感图像上取得极高的召回率和精确率(超过90\%),通过诸如特征金字塔FPN\cite{linFeaturePyramidNetworks2017}和卷积注意力CBAM\cite{wooCBAMConvolutionalBlock2018}等技巧,能够提升检测网络对小尺寸陨石坑的检测表现,并在不同的光照条件下取得稳定的结果。但是,现有模型仍然对小陨石坑缺乏敏感性。

遥感正射图像坐标系通过坐标系变换和投影变换可转换至月球经纬度坐标,因此利用转换公式,可以将不同遥感图像上的陨石坑统一至相同的经纬坐标系下,借助边界框非极大抑制算法,合并相同位置处重复出现的陨石坑检测结果,由此可得到全体候选陨石坑组成的候选陨石坑目录,需要对候选陨石坑目录作检验筛选,排除其中错误的陨石坑,才能得到最终的陨石坑目录输出。

现有的陨石坑检验方案中,较为可靠、应用较多的是专家检验,例如用于陨石坑检测与检验工具CraterTools\cite{kneisslMapprojectionindependentCraterSizefrequency2011},能够在人工标注选点基础上计算陨石坑圆参数,并自动折算至经纬坐标系下,此外还能通过使用不同投影方法和多点测量验证陨石坑尺寸测量结果。基于CraterTools,结合遥感光学图像与遥感DEM,Head\cite{headGlobalDistributionLarge2010}、Povilaitis\cite{povilaitisCraterDensityDifferences2018}、Robbins\cite{robbinsNewGlobalDatabase2019}和Yang\cite{yangLunarImpactCrater2020}等人提出了全月陨石坑目录;Jia\cite{jiaCatalogueImpactCraters2020}、Qian\cite{qianCopernicanaged200Ma2021}和Bo\cite{boCatalogueMeterscaleImpact2022}等人提出了嫦娥五号附近的显著型和小型陨石坑目录。专家检验费时费力,难以保证复现一致性。Wang\cite{wangImprovedGlobalCatalog2021}等人利用陨石坑地形数据设计验证算法,能够有效提升检验效率并保证可靠性。但是,这种方法不能筛去错误的陨石坑,且在边缘破碎的风化陨石坑处容易产生异常结果。
本文需要解决检测与检验两方面问题的自动化,同时保证检验算法的可靠性。在检测上,采用目前先进的Co-DETR算法输出边界框检测结果。在检验上,事先将DEM与遥感光学图像配准,从Bo\cite{boCatalogueMeterscaleImpact2022}等人提供的小型陨石坑目录中获取单个陨石坑的三维点,设计检验判别网络DNet,将判别网络输出的判别概率和检测网络输出的置信度以贝叶斯公式合并后得到置信概率,并由此将全部候选陨石坑检验后取得陨石坑目录。本文的工作贡献主要有:
\begin{itemize}
  \item 将CO-DETR用于陨石坑检测,在现有陨石坑检测数据集上取得最佳表现。
  \item 提出了一种基于地形数据的非极大抑制算法,精确确定候选陨石坑尺度与位置。
  \item 提供了嫦娥五号着陆点附近详细的中小型陨石坑目录。
\end{itemize}