\section{Related Works}
\label{sec:related_works}
正射遥感图像上的陨石坑检测多采用计算机视觉领域的目标检测算法。按检测头网络的构成基本单元,可分为基于CNN的检测网络和基于Transformer的检测网络。
\subsection{陨石坑检测}
\subsubsection{基于CNN}
主流的CNN检测网络以RCNN和YOLO为代表,其中以Faster RCNN\cite{renFasterRCNNRealtime2015}为代表的双阶段网络因其具备的区域建议网络RPN,在陨石坑检测任务精确率、召回率表现上更佳。Wang\cite{wangCraterIDNetEndendFully2018}等人将Faster RCNN的全连接回归头修改为全卷积回归头,得到CraterIDNet,该网络减少了参数量,但是缺乏对小尺度陨石坑的检测能力。Chen\cite{chenCraterDetectionRecognition2021}等人在CraterIDNet基础上增加FPN,增强了尺度敏感性,但是需要依赖稠密的锚点,训练过程较为复杂。Liu\cite{liuIdentificationLunarCraters2024}和Miao\cite{miaoLCDNetInnovativeNeural2024}等人不修改Faster RCNN的结构,分别使用基于小样本学习和自适应NMS的训练策略,利用FPN结构增强小尺寸陨石坑检测能力,实现在不同数据集条件下的学习和复用,但是实验表明Faster RCNN模型泛化能力不佳,在与源数据集差异较大的测试集上检测召回率很低。Yang\cite{yangLunarImpactCrater2020}和Yang\cite{yangCraterDANetConvolutionalNeural2022}等人考虑使用迁移学习和Cycle-GAN\cite{zhuUnpairedImageimageTranslation2017}跨域学习增强泛化能力,但是无疑使训练策略异常复杂,不利用于实际使用。
单阶段网络如YOLO系列、SSD等模型使用预设的锚点代替了RPN,提升了实时检测能力,但是在小尺寸陨石坑检测上表现不佳。Dai\cite{daiBoostingCraterDetection2023}等人提出用ViT\cite{dosovitskiyImageWorth16x162021}融合两种数据模态的YOLOv7网络,该网络改善了单模态检测表现,但是在整体表现上仍然弱于双阶段网络。Liu\cite{liuTwostageAdaptiveNetwork2024}等人结合归一化自适应(Normalization-Based Attention Module)、双阶段监督和跨域训练增强YOLOv5,改善了网络精确率,但是召回率大幅下降。此外另有多位学者直接应用YOLOv3\cite{fairweatherAutomaticMappingSmall2022}、YOLOv8\cite{wangCatalogueImpactCraters2024}和SSD\cite{silvestriniOpticalNavigationLunar2022}进行陨石坑检测,均在小尺寸陨石坑上表现不佳。基于热力图的无锚点检测网络CenterNet\cite{zhangAutomaticDetectionSmallscale2024,yangTopographicKnowledgeawareNetwork2024}尚不及有锚点的单阶段和双阶段网络。
\subsubsection{基于Transformer}
构成CNN网络的核心运算卷积,仅具备有限的感受野,其对图像的整体把握能力不足。Transformer提出的注意力机制\cite{vaswaniAttentionAllYou2017}弥补了这一缺点,综合CNN和Transformer搭建的DETR\cite{carionEndtoendObjectDetection2020}开创端到端检测的全新领域,但是DETR难以收敛、并未在性能上显著超越CNN基础的检测模型,仍然有较大的提升空间。因此,Deformable DETR\cite{zhuDeformableDETRDeformable2021}用弹性注意力机制取代基于MLP的注意力机制,引入类似锚点机制的焦点偏离,加快了检测网络的收敛。DINO\cite{zhangDINODETRImproved2023}在DAB-DETR\cite{liuDABDETRDynamicAnchor2022}提出的锚点查询框和DN-DETR\cite{liDNDETRAccelerateDETR}提出的带噪训练策略基础上,将解码器层的级联关系由一次回顾修改s为两次回顾,进一步提升了模型的表现性能和收敛速度。针对训练中正样本数量远远少于负样本,导致注意力学习缓慢的问题,CO-DETR\cite{zongDETRsCollaborativeHybrid2023}加入多组协作分支,增加编码器中产生的正样本查询值,有效提升了DETR系列检测模型的表现能力。鉴于DETR系列模型在目标检测领域取得先进水平,Crater-DETR\cite{guoCraterDETRNovelTransformer2024}在RT-DETR\cite{zhaoDETRsBeatYOLOs2024}结构基础上增加了噪声训练和小样本IoU损失函数,显著改善了小陨石坑的检测能力,是目前首个将DETR模型用于陨石坑检测的算法,但是在召回率表现上仍然有提升空间。
\subsection{后处理算法}
检测算法所得候选陨石坑需经过后处理算法才能形成陨石坑目录。常用的后处理算法包括非极大抑制、检验确认等。
\subsubsection{非极大抑制}
普通NMS采用简单的IoU计算,保留相同标签下最高置信度的边界框,但是其存在诸如误删、耗时以及IoU阈值不易确定等问题。Soft NMS\cite{bodlaSoftNMSImprovingObject2017}用指数衰减被抑制候选框的置信分数,Softer NMS\cite{heSofterNMSRethinkingBounding2018}在其基础上加入增加对定位置信度的计算、以对应顶点加权合并作为抑制结果,提升了边界框的定位精度,但是Softer NMS需要修改检测头的回归输出,在训练阶段需要回传梯度。Adaptive NMS\cite{liuAdaptiveNMSRefining2019}针对不同的物体密度自适应设计抑制阈值,改善了稠密重叠物体的检测表现,Adaptive NMS也需要训练。针对NMS耗时较长的问题,Matrix NMS\cite{wangSOLOSimpleFramework2021}用矩阵乘法替代循环加快了计算过程,但是需要占用较多的存储资源。由于陨石坑具备圆形外表,Liu\cite{liuIdentificationLunarCraters2024}等人用圆形框取代方形框计算IoU,但是在重叠陨石坑条件下存在误删情况。Miao\cite{miaoLCDNetInnovativeNeural2024}等人将Adaptive NMS应用于重叠陨石坑的检测,减少了重叠情况对NMS的影响。
\subsubsection{检验确认}
大多数候选陨石坑经检验确认后可用于制作目录,现有检验算法大多采用专家手工检验。其中,Yang\cite{yangLunarImpactCrater2020}等人仅在遥感光学图像上确认;Head\cite{headGlobalDistributionLarge2010}、Povilaitis\cite{povilaitisCraterDensityDifferences2018}、Robbins\cite{robbinsNewGlobalDatabase2019}、Jia\cite{jiaCatalogueImpactCraters2020}、Qian\cite{qianCopernicanaged200Ma2021}和Bo\cite{boCatalogueMeterscaleImpact2022}等人结合地形数据和光学图像共同确认。目前仅有少数学者研究使用算法进行陨石坑检验确认,Wang\cite{wangImprovedGlobalCatalog2021}等人以候选陨石坑中心沿八个等分方向取最高点作为真实边缘点重新拟合圆,以此作为陨石坑真实边缘,该算法仅考虑了单个候选陨石坑的地形数据,不能排除错误的检测结果。